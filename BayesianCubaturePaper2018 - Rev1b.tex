% This is a general template file for the LaTeX package SVJour3
% for Springer journals. Original by Springer Heidelberg, 2010/09/16
%
% Use it as the basis for your article. Delete % signs as needed.
%
% This template includes a few options for different layouts and
% content for various journals. Please consult a previous issue of
% your journal as needed.
%
\RequirePackage{amsmath}
\RequirePackage{fix-cm}
%
%\documentclass{svjour3}                     % onecolumn (standard format)
%\documentclass[smallcondensed]{svjour3}     % onecolumn (ditto)
%\documentclass[smallextended]{svjour3}       % onecolumn (second format)
\documentclass[twocolumn]{svjour3}          % twocolumn
%
\smartqed  % flush right qed marks, e.g. at end of proof
%
\usepackage{graphicx}
\usepackage{amssymb}
%
% insert here the call for the packages your document requires
%\usepackage{mathptmx}      % use Times fonts if available on your TeX system
%\usepackage{latexsym}
% etc.
%
% Jag's 
\usepackage{tabu}
\usepackage{cancel}
\usepackage{algorithm}
\usepackage{algorithmicx}
\usepackage{algpseudocode}
%\usepackage[caption=false]{subfig}
\usepackage{subcaption}
\usepackage{booktabs, mathtools}
\usepackage[dvipsnames]{xcolor}
\usepackage{url}
\usepackage[titletoc,title]{appendix}
%\usepackage{refcheck}  % points out unused labels, refs ...

\captionsetup{compatibility=false}

\algdef{SE}[DOWHILE]{Do}{doWhile}{\algorithmicdo}[1]{\algorithmicwhile\ #1}%

\DeclareMathOperator{\Order}{{\mathcal O}}

% please place your own definitions here and don't use \def but
% \newcommand{}{}
\newtheorem{prop}{Proposition}
\providecommand{\HickernellFJ}{Hickernell}
\newcommand{\bm}[1]{\boldsymbol{#1}}
\newcommand{\mSigma}{\mathsf{\Sigma}}
\newcommand{\mB}{\mathsf{B}}
\newcommand{\smallocite}[1]{{\small\ocite{#1}}}
% \newcommand{\bm}[1]{\boldsymbol{#1}}
\newcommand{\dif}[1]{\text{d}{#1}}
\newcommand{\D}[1]{\text{d}{#1}}
\newcommand{\trace}[1]{\textup{trace}{#1}}

\newcommand{\naturals}{\mathbb{N}}
\newcommand{\natzero}{\mathbb{N}_0}
\newcommand{\reals}{\mathbb{R}}
\newcommand{\integers}{\mathbb{Z}}
\newcommand{\posIntegers}{\mathbb{Z}_{> 0}}
\newcommand{\complex}{\mathbb{C}}
\newcommand{\hilbert}{\mathbb{H}}
\newcommand{\Ex}{\mathbb{E}}

\newcommand{\cf}{\mathcal{F}}
\newcommand{\cx}{\mathcal{X}}
\newcommand{\tcx}{\widetilde{\cx}}
\newcommand{\rC}{\mathring{C}}
\newcommand{\ry}{\mathring{y}}
\newcommand{\rlambda}{\mathring{\lambda}}

\newcommand{\valpha}{{\bm{\alpha}}}
\newcommand{\vbeta}{{\bm{\beta}}}
\newcommand{\vDelta}{{\boldsymbol{\Delta}}}
\newcommand{\vlambda}{{\bm{\lambda}}}
\newcommand{\vphi}{{\bm{\phi}}}
\newcommand{\vpsi}{{\bm{\psi}}}
\newcommand{\vtheta}{{\bm{\theta}}}
\newcommand{\vzeta}{{\bm{\zeta}}}
\newcommand{\vthetaMLE}{\bm{\theta}_{\text{MLE}}}
\newcommand{\hvtheta}{\hat{\vtheta}}
\newcommand{\va}{\bm{a}}
\newcommand{\vA}{\bm{A}}
\newcommand{\vb}{\bm{b}}
\newcommand{\vc}{\bm{c}}
\newcommand{\vC}{\bm{C}}
\newcommand{\tvc}{\tilde{\bm{c}}}
\newcommand{\vg}{\bm{g}}
\newcommand{\vh}{\bm{h}}
\newcommand{\vf}{\bm{f}}
\newcommand{\vk}{\bm{k}}
\newcommand{\vm}{\bm{m}}
\newcommand{\vs}{\bm{s}}
\newcommand{\vt}{\bm{t}}
\newcommand{\vv}{\bm{v}}
\newcommand{\vV}{\bm{V}}
\newcommand{\vw}{\bm{w}}
\newcommand{\vW}{\bm{W}}
\newcommand{\vx}{\bm{x}}
\newcommand{\dx}{\dif{{x}}}
\newcommand{\dt}{\dif{{t}}}
\newcommand{\dvx}{\dif{\bm{x}}}
\newcommand{\dvs}{\dif{\bm{s}}}
\newcommand{\dvt}{\dif{\bm{t}}}
\newcommand{\vrho}{\bm{\rho}}
\newcommand{\hy}{\hat{y}}
\newcommand{\vy}{\bm{y}}
\newcommand{\vY}{\bm{Y}}
\newcommand{\hvy}{\hat{\vy}}
\newcommand{\vz}{\bm{z}}
\newcommand{\vZ}{\bm{Z}}
\newcommand{\dvz}{\dif{\bm{z}}}
\newcommand{\tf}{\tilde{f}}
\newcommand{\vPsi}{\boldsymbol{\Psi}}

\newcommand{\tvv}{\tilde{\vv}}
\newcommand{\tvz}{\tilde{\vz}}

\newcommand{\vCvtheta}{{C_\vtheta}}
\newcommand{\hc}{\widehat{c}}

\newcommand{\hatvy}{\hat{\bm{y}}}
\newcommand{\haty}{\hat{y}}
\newcommand{\tvy}{\tilde{\bm{y}}}
\newcommand{\ty}{\tilde{y}}
\newcommand{\vzero}{\bm{0}}
\newcommand{\vone}{\bm{1}}
\newcommand{\tvone}{\tilde{\bm{1}}}
\newcommand{\mA}{\mathsf{A}}
\newcommand{\mC}{\mathsf{C}}
\newcommand{\rmC}{\mathring{\mathsf{C}}}
\newcommand{\mCtheta}{{\mathsf{C}_{\vtheta}}}
%\newcommand{\mCthetaInv}{{\mathsf{C}^{-1}_{\vtheta}}}
%\newcommand{\mCthetaMLE}{{\mathsf{C}_{\vthetaMLE}}}
%\newcommand{\mCthetaInvMLE}{{\mathsf{C}^{-1}_{\vthetaMLE}}}
\newcommand{\mCInv}{\mathsf{C}^{-1}}
\newcommand{\cov}{{\textup{cov}}}
\newcommand{\var}{{\textup{var}}}


\newcommand{\tmC}{\widetilde{\mathsf{C}}}
\newcommand{\tlambda}{\tilde{\lambda}}

\newcommand{\mL}{\mathsf{L}}

\newcommand{\mLambda}{\mathsf{\Lambda}}
\newcommand{\mLambdaInv}{\mathsf{\Lambda}^{-1}}

\newcommand{\mV}{\mathsf{V}}
\newcommand{\mW}{\mathsf{W}}

\newcommand{\calN}{\mathcal{N}}
\newcommand{\me}{\mathrm{e}}

\newcommand{\tvrho}{\widetilde{\vrho}}
\newcommand{\heta}{\hat{\eta}}
\newcommand{\hmu}{\widehat{\mu}}
\newcommand{\hsigma}{\widehat{\sigma}}
\newcommand{\hnu}{\hat{\nu}}
\newcommand{\rhoCond}{\mathring{\vrho}}

\newcommand{\MVN}{\textup{MVN}}
\newcommand{\MLE}{\textup{EB}}
\newcommand{\full}{\textup{full}}
\newcommand{\GCV}{\textup{GCV}}
%\newcommand{\errtol}{\text{tol}}
\newcommand{\errtol}{\varepsilon}
\newcommand{\errn}{\text{err}_{n}}
\newcommand{\diag}{\text{diag}}
\newcommand{\err}{\textup{err}}
\newcommand{\code}[1]{\texttt{#1}}

\def\abs#1{\ensuremath{\left \lvert #1 \right \rvert}}
\newcommand{\norm}[2][{}]{\ensuremath{\left \lVert #2 \right \rVert}_{#1}}
\newcommand{\ip}[3][{}]{\ensuremath{\left \langle #2, #3 \right \rangle_{#1}}}

\newenvironment{nalign}{
    \begin{equation}
    \begin{aligned}
}{
    \end{aligned}
    \end{equation}
    \ignorespacesafterend
}

\providecommand{\argmin}{\operatorname*{argmin}}
\providecommand{\argmax}{\operatorname*{argmax}}
\newcommand\figref{Figure~\ref}
\newcommand\secref{Section~\ref}

\graphicspath{{.}{./figures/}{D:/Mega/MyWriteupBackup/Sep_2ndweek_1/}}
%\graphicspath{{./figures/}}

%
% Insert the name of "your journal" with
\journalname{Automatic Bayesian Cubature}
%

\newcommand{\FJHNote}[1]{{\textcolor{blue}{FJH: #1}}}
\newcommand{\JRNote}[1]{{\textcolor{green}{JR: #1}}}


%\allowdisplaybreaks[4]
\begin{document}
\setlength\abovedisplayskip{1pt}
\setlength{\belowdisplayskip}{1pt}

\title{Fast Automatic Bayesian Cubature Using Lattice Sampling
%\thanks{}
}
% Grants or other notes about the article that should go on the front
% page should be placed within the \thanks{} command in the title
% (and the %-sign in front of \thanks{} should be deleted)
%
% General acknowledgments should be placed at the end of the article.

%\subtitle{Do you have a subtitle?\\ If so, write it here}

%\titlerunning{Short form of title}        % if too long for running head

\author{R. Jagadeeswaran         \and
        Fred J. Hickernell %etc.
}

%\authorrunning{Short form of author list} % if too long for running head

\institute{R. Jagadeeswaran \at
              Department of Applied Mathematics, \\
              Illinois Institute of Technology \\
              10 W. 32nd St., Room 208\\
              Chicago IL 60616\\
              \email{jrathin1@iit.edu}           %  \\
%             \emph{Present address:} of F. Author  %  if needed
           \and
           Fred J. Hickernell \at
           Center for Computational Science and Department of Applied Mathematics, 
           Illinois Institute of Technology \\
           PS 106, 3105 S. Dearborn St., 
           Chicago IL 60616
           \\
           \email{hickernell@iit.edu} 
}

\date{Received: date / Accepted: date}
% The correct dates will be entered by the editor

\maketitle

\begin{abstract}
Automatic cubatures approximate multidimensional integrals to user-specified error tolerances.  For high dimensional problems it is difficult to adaptively change the sampling pattern, but one can automatically determine the
sample size, $n$, give a fixed and reasonable sampling pattern. We take this approach here using a Bayesian perspective.  We postulate that the integrand is an instance of a Gaussian stochastic process parameterized by a constant mean and a covariance function defined by a scale parameter times a parameterized function specifying how the integrand values at two different points in the domain are related.
These hyperparameters are handled using integrand values via empirical Bayes, full Bayes, and generalized cross-validation.  The sample size, $n$, is chosen to make the half-width of the credible interval for the Bayesian posterior mean no greater than the error tolerance. 

The process outlined above typically requires a computational cost of $O(N_{\text{opt}}n^3)$, where $N_{\text{opt}}$ is the number of optimizations required to identify the hyperparameters. Our innovation is to pair low discrepancy nodes with matching kernels that lower the  computational cost to $O(N_{\text{opt}} n \log n)$.   This approach is demonstrated using rank-1 lattice sequences and shift-invariant kernels.  Our algorithm is  implemented in the Guaranteed Automatic Integration Library (GAIL).


\keywords{Bayesian cubature \and Fast automatic cubature \and GAIL \and Probabilistic numeric methods }
% \PACS{PACS code1 \and PACS code2 \and more}
% \subclass{MSC code1 \and MSC code2 \and more}
\end{abstract}

\section{Introduction}
\label{intro}
Cubature is the problem of inferring a numerical value for an integral, 
$ \mu := \int_{\reals^d} g(\vx) \, \dif \vx$, where $\mu$ has no closed form analytic expression. Typically, $g$ is accessible as a black-box algorithm. 
Cubature is a key component of many problems in scientific computing, finance, statistical modeling, and machine learning.  

The integral may often be expressed as
\begin{align}
\label{eqn:defn_mu}
\mu:= \Ex[f(\boldsymbol{X})] = \int_{[0,1]^d} f(\vx)\, \dif\vx, 
\end{align}
where $f:[0,1]^d \to \reals$ is the integrand, and $\boldsymbol{X} \sim \mathcal{U}[0,1]^d$.  The process of transforming the original integral into the form of \eqref{eqn:defn_mu} is not addressed here. See \cite[Section 2.11]{DicEtal14a} 
for a discussion of variable transformations. The cubature may be an affine function of integrand values:
\begin{align}
\label{eqn:defn_hmu}  % remove this
\hmu := w_0 + \sum_{i=1}^{n} f(\vx_i) w_i,
\end{align}
where the weights, $w_0$, and  $\vw = (w_i)_{i=1}^n \in \reals^n$, and the nodes, $\{\vx_i\}_{i=1}^n \subset [0,1]^d$, are chosen to make the error, $\abs{\mu - \hmu}$, small. The integration domain $[0,1]^d$ is convenient for the low discrepancy node sets \cite{DicEtal14a,SloJoe94} that we use.  The nodes are assumed to be deterministic.

The users of cubature algorithms typically want the error to be no greater than their error tolerance denoted by $\varepsilon$, i.e., 
\begin{align}
\label{eqn:err_crit} 
\abs{\mu - \hmu} \leq \errtol .
\end{align}
Some stopping criteria used in practice are heuristic.  Other rigorous algorithms require rather strong assumptions about the integrand, such as an upper bound on its variance (for simple Monte Carlo) or total variation (for quasi-Monte Carlo).  We take a Bayesian approach by constructing a stopping criterion that is based on a credible interval.  We build upon the work of Briol et al.~\cite{BriEtal18a}, Diaconis~\cite{Dia88a}, O'Hagan~\cite{OHa91a}, Ritter~\cite{Rit00a}, Rasmussen and Ghahramani~\cite{RasGha03a}, and others.  Our algorithm is an example of \emph{probabilistic numerics}.

The primary contribution of this article is to demonstrate how the choice of a family of covariance kernels that match the low discrepancy sampling nodes facilitates fast computation of the cubature and the data-driven stopping criterion.  Our Bayesian cubature requires a computational cost of
\begin{equation} \label{eqn:OuralgoCost}
    \Order\bigl(N_{\textup{opt}}[n \$(f) + n\$(C) + n \log(n)] \bigr),
\end{equation} 
where $\$(f)$ is the cost of one integrand value, $\$(C)$ is the cost of a single covariance kernel value,  $\Order(n \log(n))$ is the cost of a fast Fourier transform, and $N_{\textup{opt}}$ is an upper bound on the number of optimization steps required to choose the hyperparameters. If function evaluation is expensive, such as coming from computationally intensive simulation, or $\$(f) = \Order(d)$ for large $d$, then $\$(f)$ might be similar in magnitude to $\log(n)$ in practice.  Typically, $\$(C) = \Order(d)$.  Note that the $\Order(n \log(n))$ contribution is $d$ independent.

By contrast to our fast algorithm, the typical computational cost for Bayesian cubature is
\begin{equation} \label{eqn:TheiralgoCost}
    \Order\bigl(N_{\textup{opt}}[n \$(f) + n^2\$(C) + n^3] \bigr),
\end{equation} 
which is explained in Section \ref{sec:bayes_cubature_algo}. Note that the work beyond evaluating the integrand in \eqref{eqn:TheiralgoCost} is much larger than that in \eqref{eqn:OuralgoCost}.  

Hickernell~\cite{Hic17a} compares different approaches to cubature error analysis depending on whether the rule is deterministic or random and whether the integrand is assumed to be deterministic or random.  Error analysis that assumes a deterministic integrand lying in a Banach space leads to an error bound that is typically impractical for deciding how large $n$ must be to satisfy \eqref{eqn:err_crit}.  The deterministic error bound includes a (semi-) norm of the integrand, often called the variation, which is often more complex to compute than the original integral.

Hickernell and Jim\'enez-Rugama~\cite{HicJim16a,JimHic16a} have developed stopping criteria for cubature rules based on low discrepancy nodes by tracking the decay of the discrete Fourier coefficients of the integrand.  The algorithm proposed here also relies on discrete Fourier coefficients, but in a different way.  Although we only explore automatic Bayesian cubature for absolute error tolerances, the recent work by Hickernell, Jim\'enez-Rugama, and Li~\cite{HicEtal17a} suggests how one might accommodate more general error criteria, such as relative error tolerances.

Section \ref{sec:BC} explains the Bayesian approach to estimate the posterior cubature error and defines our automatic Bayesian cubature. Although much of this material is known, it is included for completeness.  We end Section \ref{sec:BC}  by demonstrating why Bayesian cubature is typically computationally expensive.
Section \ref{sec:fast_BC}  introduces the concept of covariance kernels that match the nodes and expedite the computations required by our automatic Bayesian cubature. 
Section \ref{sec:shift_invariant_kernel} implements this concept for shift invariant kernels and rank-1 lattice nodes.  It also describes how to avoid cancellation error for kernels of product form.  Numerical examples are provided in Section \ref{sec:NumExp} to demonstrate our new algorithm.  We conclude with a brief discussion.



\section{Bayesian Cubature} \label{sec:BC} 


\subsection{Bayesian posterior error}
\label{sec:BayesPostErr}

Suppose that the integrand, $f$, is assumed to be an instance of a Gaussian stochastic process \cite{BriEtal18a}, \cite{Dia88a}, \cite{OHa91a}, \cite{Rit00a}, \cite{RasGha03a}, i.e., $f \sim \mathcal{GP}(m,s^2 C_\vtheta)$.  Specifically, $f$ is a random function with real-valued constant mean $m$ and covariance function $s^2C_\vtheta$:
\begin{gather*}
        m = \Ex[f(\vx)] \qquad \forall \vx \in \reals^d, \\
        \Ex\{[f(\vt) - m][f(\vx) -m]\} = s^2 C_\vtheta(\vt,\vx) \qquad \forall \vt, \vx \in \reals^d.
\end{gather*}
Here $s$ is a non-negative scale factor, and $C_\vtheta: [0,1]^d \times [0,1]^d \to \mathbb{R} $ is a symmetric, positive-definite function and parameterized by $\vtheta$:
\begin{multline} \label{FJH:eq:CondPosDef}
\mC_\vtheta^T = \mC_\vtheta,  \; \va^T \mC_\vtheta \va > 0, \; \text{where }  \mC_\vtheta = \left(  C_\vtheta(\vx_i,\vx_j)  \right)_{i,j=1}^n,\\
 \forall \va \ne 0, \;
 n\in \mathbb{N}, \; \text{ distinct }\vx_1, \ldots, \vx_n \in [0,1]^d.
\end{multline}
Procedures for estimating or integrating out the hyperparameters $m$, $s$, and $\vtheta$ are explained later in this section.

Furthermore, for a Gaussian process, all vectors of linear functionals of $f$ have a multivariate Gaussian distribution. For any deterministic sampling scheme with distinct nodes, $\{\vx_i\}_{i=1}^n$, and defining  $\vf  := \left( f(\vx_i)\right)_{i=1}^n$ as the multivariate normal vector of function values, it follows from the definition of a Gaussian process that 
\begin{subequations} \label{eqn:fGaussDist}
\begin{align}
\vf  & \sim \calN(m \vone, s^2 \mCtheta), \\
\nonumber & \qquad \qquad \text{where $\vone$ is a vector of all ones,}\\
\mu & \sim \calN(m, s^2 c_{0\vtheta}), 
\\
\text{where }
c_{0\vtheta} &:= \int_{[0,1]^{d}\times [0,1]^{d}} C_\vtheta(\vt,\vx) \, \dif{\vt} \, \dif{\vx}, \\
\cov(\vf, \mu) &= \left(  \int_{[0,1]^d} C_\vtheta(\vt,\vx_i) \, \D \vt \right)_{i=1}^n  =: \vc_\vtheta.
\end{align}
\end{subequations}
Again, $c_{0\vtheta}$ and $\vc_\vtheta$ depend explicitly on $\vtheta$.  We assume that the integrals in these definitions can be computed analytically.  We need the following lemma pertaining to a conditional normal distribution to derive the distribution of the posterior error of our cubature. 

\begin{lemma} \cite[(A.6), (A.11--13)]{RasWil06a} \label{thrm:condDist} If $\vY = (\vY_1, \vY_2)^T \sim \calN (\vm,\mSigma)$, where $\vY_1$ and $\vY_2$ are random vectors of arbitrary length, and 
	\begin{gather*}
	\vm = \begin{pmatrix} \vm_1 \\ \vm_2 \end{pmatrix} = \begin{pmatrix} \Ex(\vY_1) \\ \Ex(\vY_2) \end{pmatrix}, \\
	\mSigma = \begin{pmatrix}
	\mSigma_{11} & \mSigma_{21}^T \\ 	\mSigma_{21} & \mSigma_{22}
	\end{pmatrix} =
	\begin{pmatrix}
	\var(\vY_{1}) & \cov(\vY_{1}, \vY_2) \\ 	\cov(\vY_2,\vY_{1}) & \var(\vY_{2})
	\end{pmatrix} 
	\end{gather*}
	then 
	\begin{multline*}
	\vY_1 \vert \vY_2 \; \sim \; \calN \bigl(\vm_1 + \mSigma_{21}^T \mSigma_{22}^{-1}(\vY_2 - \vm_2), \\ \mSigma_{11} - \mSigma_{21}^T \mSigma_{22}^{-1} \mSigma_{21} \bigr).
	\end{multline*}
Moreover, the inverse of the matrix $\mSigma$ may be partitioned as
\begin{gather*}
\mSigma^{-1} = \begin{pmatrix} \mA_{11} & \mA_{21}^T \\ \mA_{21} & \mA_{22} \end{pmatrix}, \\
\mA_{11} = (\mSigma_{11} - \mSigma_{12} \mSigma_{22}^{-1} \mSigma_{21})^{-1}, \qquad 
\mA_{21} = -  \mSigma_{22}^{-1} \mSigma_{21} \mA_{11}, \\ 
\mA_{22} = \mSigma_{22}^{-1} + \mSigma_{22}^{-1} \mSigma_{21} \mA_{11} \mSigma_{21}^T \mSigma_{22}^{-1}.
\end{gather*}

\end{lemma}


It follows from Lemma \ref{thrm:condDist} that the \emph{conditional} distribution of the integral given observed function values, $\vf = \vy$ is also Gaussian:
\begin{multline} \label{eqn:condInteg}
\mu | (\vf = \vy) \sim \calN \bigl(m (1 - \vc_\vtheta^T \mCtheta^{-1} \vone)  + \vc_\vtheta^T \mCtheta^{-1} \vy, 
\\
s^2(c_{0\vtheta}  -\vc_\vtheta ^T \mCtheta^{-1} \vc_\vtheta) \bigr).
\end{multline}
The natural choice for  the cubature is the posterior mean of the integral, namely, 
\begin{align}
\label{eqn:BayesCub}
\widehat{\mu}  \vert ( \vf = \vy)
= m(1 - \vc_\vtheta^T  \mCtheta^{-1} \vone)
+ \vc_\vtheta^T \mCtheta^{-1} \vy,
\end{align}
which takes the form of \eqref{eqn:defn_hmu}.
Under this definition, the cubature error has zero mean and a variance depending on the choice of nodes:
\begin{align*}
%\label{eqn_error_cond_prob}
(\mu-\hmu) | (\vf = \vy)
 \sim  \calN 
\left(
0, \;
s^2 (c_{0\vtheta} - \vc_\vtheta^T\mCtheta^{-1}\vc_\vtheta) 
\right).
\end{align*}
A credible interval for the integral is given given by 
\begin{subequations} \label{eqn_prob_confidence_interval}
\begin{gather}
\mathbb{P}_f \left[
|\mu-\hmu| \leq \err_{\textup{CI}}
\right] = 99\%, \\
\err_{\textup{CI}} = 2.58 s \sqrt{c_{0\vtheta} - \vc_\vtheta^T\mCtheta^{-1}\vc_\vtheta}
\end{gather}
\end{subequations}
Naturally, $2.58$ and $99\%$ can be replaced by other quantiles and credible levels.

%%%%%%%%%%%%%%%%%%%%%%%%%%%%%%%%%%%%%%%%%%%%%%%%%%%%%%%%%
%%%%%%%%%%%%%%%%%%%%%%%%%%%%%%%%%%%%%%%%%%%%%%%%%%%%%%%%%
\subsection{Parameter estimation}
%%%%%%%%%%%%%%%%%%%%%%%%%%%%%%%%%%%%%%%%%%%%%%%%%%%%%%%%%
%%%%%%%%%%%%%%%%%%%%%%%%%%%%%%%%%%%%%%%%%%%%%%%%%%%%%%%%%
The credible interval in \eqref{eqn_prob_confidence_interval} suggests how our automatic Bayesian cubature proceeds.  Integrand data is accumulated until the width of the credible interval, $\err_{\textup{CI}}$, is no greater than the error tolerance.  As $n$ increases, one expects $\sqrt{c_{0\vtheta} - \vc_\vtheta^T\mCtheta^{-1}\vc_\vtheta}$ to decrease for well-chosen nodes, $\{\vx_i\}_{i=1}^n$.

Note that $\err_{\textup{CI}}$ has no explicit dependence on the integrand values, even though one would intuitively expect that larger integrand should imply a  larger $\err_{\textup{CI}}$.  This is because parameters, $m, s$, and $\vtheta$, have not yet been inferred from integrand data.  After inferring the parameters, $\err_{\textup{CI}}$ does reflect the size of the integrand values. This section describes three approaches to parameter estimation.

\begin{theorem} \label{thm:param} There are at least three approaches to estimating or integrating out the hyperparameters defining the Gaussian process from which the integrand is drawn: empirical Bayes, full Bayes, and generalized cross validation.  Under these three approaches we have the following:
\begin{allowdisplaybreaks}
\begin{align}
    \label{eqn_m_MLE}
m_\MLE &= \frac{\vone^T \mCInv_\vtheta \vy }{ \vone^T \mCInv_\vtheta \vone}, \qquad
m_{\textup{GCV}} = \frac{\vone^T \mC_\vtheta^{-2} \vy}{\vone^T \mC_\vtheta^{-2} \vone}, \\
\label{eqn_s2_MLE}
s^2_{\MLE} 
&= 
\frac{1}{n}
\vy^T 
\left[ \mCInv_\vtheta - 
\frac{ \mCInv_\vtheta \vone \vone^T \mCInv_\vtheta }{\vone^T\mCInv_\vtheta \vone}
\right] \vy, \\
\nonumber
\hsigma_{\textup{full}}^2 
& = \frac{1}{n-1}
\vy^T\left[ \mC_\vtheta^{-1} 
- \frac{ \mC_\vtheta^{-1} \vone\vone^T \mC_\vtheta^{-1}}{\vone^T \mC_\vtheta^{-1} \vone}  \right]\vy
\\ 
& \times  \left[\frac{(1 - \vc_{\vtheta}^T \mC_\vtheta^{-1} \vone)^2}{\vone^T \mC_\vtheta^{-1} \vone} + (c_{0\vtheta}  -\vc_{\vtheta} ^T \mC_\vtheta^{-1} \vc_{\vtheta}) \right], \\
\nonumber
 s^2_{\textup{GCV}} & = \vy^T \left[\mC_\vtheta^{-2} - \frac{\mC_\vtheta^{-2} \vone \vone^T \mC_\vtheta^{-2}}{\vone^T \mC_\vtheta^{-2} \vone}  \right] \vy  \left[ \trace(\mC_\vtheta^{-1}) \right]^{-1}, \\
\nonumber
\vtheta_\MLE
&= \argmin_{\vtheta} \biggl \{
\log\left(\vy^T 
\left[ \mC_\vtheta^{-1} - 
\frac{ \mCInv_\vtheta \vone \vone^T \mCInv_\vtheta }{\vone^T\mCInv_\vtheta \vone}
\right] \vy 
\right)  \\
\label{eqn:thetaMLE}
 & \qquad +  \frac{1}{n} \log(\det(\mC_\vtheta))
\biggr \}, \\
\nonumber
\vtheta_{\textup{GCV}} &= \argmin_\vtheta \biggl\{\log \left(  \vy^T \left[\mC^{-2}_\vtheta - \frac{\mC^{-2}_\vtheta \vone \vone^T \mC^{-2}_\vtheta}{\vone^T \mC^{-2}_\vtheta \vone}  \right] \vy \right)  \\
\label{vthetaGCV}
& - \log \left ( \trace(\mC^{-2}_\vtheta) \right ) \biggr\}, \\
\hmu_\MLE  &= \hmu_\full =
\left(
\frac{ (1 - \vone^T  \mCInv_\vtheta\vc_{\vtheta} )  \vone }{ \vone^T \mCInv_\vtheta \vone}   +  \vc_{\vtheta} 
\right)^T  \mCInv_\vtheta \vy, \\
\label{eqn:muCV}
\widehat{\mu}_{\textup{GCV}}
& = \left(\frac{(1 - \vone^T  \mC_\vtheta^{-1}\vc_{\vtheta}) \mC_\vtheta^{-1} \vone}{\vone^T \mC_\vtheta^{-2} \vone} + \vc_{\vtheta} \right)^T \mC_\vtheta^{-1} \vy, \\
\nonumber 
\err_{\mathsf{x}} & = 2.58 s_{\mathsf{x}} \sqrt{c_{0\vtheta} - \vc_{\vtheta}^T\mC_\vtheta^{-1}\vc_{\vtheta} }, \quad \mathsf{x} \in \{\MLE, \GCV\}, \\
\nonumber 
\err_{\textup{full}} 
& = t_{n-1,0.995} \hsigma_{\textup{full}} > \err_\MLE,
\end{align}
\end{allowdisplaybreaks}
\begin{multline}
\label{eqn_prob_CI}
\mathbb{P}_f \left[
|\mu-\hmu_{\mathsf{x}}| \leq \err_{\mathsf{x}} \right]  = 99\%, \\ \mathsf{x} \in \{\MLE, \full, \GCV\}.
\end{multline}
Here $t_{n-1,0.995}$ denotes the $99.5$ percentile of a standard Student's $t$-distribution with $n-1$ degrees of freedom.
\end{theorem}

For the results summarized in the theorem above, note that if the original covariance function, $C_\vtheta$, is replaced by $b C_\vtheta$ for some positive constant $b$, the cubature, $\hmu$, the estimates of $\vtheta$, and the credible interval widths, $\err_{\mathsf{x}}$ for $\mathsf{x} \in \{\MLE, \full, \GCV\}$, all remain unchanged.  The estimates of $s^2$ are multiplied by $b^{-1}$, as would be expected. 

%%%%%%%%%%%%%%%%%%%%%%%%%%%%%%%%%%%%%%%%%%%%%%%%%%%%%%%%%
%%%%%%%%%%%%%%%%%%%%%%%%%%%%%%%%%%%%%%%%%%%%%%%%%%%%%%%%%
\subsubsection{Proof for Empirical Bayes}  \label{sec:MLE}
%%%%%%%%%%%%%%%%%%%%%%%%%%%%%%%%%%%%%%%%%%%%%%%%%%%%%%%%%
%%%%%%%%%%%%%%%%%%%%%%%%%%%%%%%%%%%%%%%%%%%%%%%%%%%%%%%%%
The empirical Bayes approach estimates the parameters, $m$, $s$, and $\vtheta$ via maximum likelihood estimation (MLE).  The log-likelihood function of the parameters given the function data $\vy$ is:
\begin{multline*}
l(s,m,\vtheta | \vy)
= -\frac{1}{2} s^{-2} (\vy-m\vone)^T\mCInv_\vtheta(\vy-m\vone) 
\\
 - \frac{1}{2} \log(\det\, \mC_\vtheta) - \frac{n}{2} \log(s^2) + \text{constants.}
\end{multline*}
Maximizing the log-likelihood first with respect to $m$ and then with respect to $s$ yields the values given in Theorem \ref{thm:param}.
To obtain $\vtheta_\MLE$, we substitute $m_\MLE$ and $s_\MLE$ into $l(s,m,\vtheta | \vy)$ and then numerically optimize:
\begin{align}
\nonumber
\vtheta_\MLE
&= \argmin_{\vtheta} \biggl \{
\log\left(\vy^T 
\left[ \mCInv_\vtheta - 
\frac{ \mCInv_\vtheta \vone \vone^T \mCInv_\vtheta }{\vone^T\mCInv_\vtheta \vone}
\right] \vy 
\right)  \\
\label{eqn:thetaMLE}
 & \qquad +  \frac{1}{n} \log(\det(\mC_\vtheta))
\biggr \}.
\end{align}
The empirical Bayes estimate of $\vtheta$ balances minimizing the covariance scale factor, $s^2_{\MLE}$, against minimizing  $\det(\mC_\vtheta)$. 
Under these estimates of the parameters, the cubature \eqref{eqn:BayesCub} and the credible interval \eqref{eqn_prob_confidence_interval} are explicitly written as in Theorem \ref{thm:param}.
The quantities $c_{0\vtheta}$, $\vc_\vtheta$, and $\mC_\vtheta$ are assumed implicitly to be based on $\vtheta = \vtheta_\MLE$.   


%%%%%%%%%%%%%%%%%%%%%%%%%%%%%%%%%%%%%%%%%%%%%%%%%%%%%%%%%
%%%%%%%%%%%%%%%%%%%%%%%%%%%%%%%%%%%%%%%%%%%%%%%%%%%%%%%%%
\subsubsection{Proof for Full Bayes} \label{sec:fullBayes}
%%%%%%%%%%%%%%%%%%%%%%%%%%%%%%%%%%%%%%%%%%%%%%%%%%%%%%%%%
%%%%%%%%%%%%%%%%%%%%%%%%%%%%%%%%%%%%%%%%%%%%%%%%%%%%%%%%%
Rather than use maximum likelihood to determine $m$ and $s$ one can treat them as hyperparameters with a non-informative, conjugate prior, namely $\vrho_{m,s^2}(\xi, \lambda) \propto 1/\lambda$. We want to compute $\rho_{\mu|\vf}(z | \vy)$, the conditional posterior density of $\mu$ given the data $\vf = \vy$.  This may be expressed as 
\begin{multline*}
    \rho_{\mu|\vf}(z | \vy) 
= \int_{0}^\infty \int_{-\infty}^\infty 
\rho_{\mu | m, s^2, \vf}(z | \xi, \lambda , \vy) \\
\times  \rho_{m, s^2 | \vf}(\xi, \lambda | \vy)  \, \D \xi \D \lambda, 
\end{multline*}
where $\rho_{m, s^2 | \vf}$ is the posterior density of the hyperparameters given the function data. Bayes Theorem tells us that $\rho_{m, s^2 | \vf} \propto \rho_{\vf | m, s^2} \, \vrho_{m,s^2}$, so 
\begin{align}
\nonumber 
    \rho_{\mu|\vf}(z | \vy) 
& = \int_{0}^\infty \int_{-\infty}^\infty 
\rho_{\mu | m, s^2, \vf}(z | \xi, \lambda , \vy) \\
\nonumber
& \qquad \times  \rho_{\vf | m, s^2} (\vy | \xi, \lambda) \, \vrho_{m,s^2}(\xi,\lambda)  \, \D \xi \D \lambda \\
\nonumber
& \propto \left( 1 +  \frac{(z - \hmu_{\MLE})^2}{(n-1) \hsigma_{\textup{full}}^2} \right)^{-n/2},
\end{align}
where $\hsigma_{\textup{full}}^2$ is given in Theorem \ref{thm:param}, and the result above is derived Appendix \ref{appendix:full_bayes}.

This means that $\mu \vert (\vf = \vy )$, properly centered and scaled, has a Student's $t$-distribution with $n-1$ degrees of freedom.   The estimated integral is the same as in the empirical Bayes case, $\hmu_{\textup{full}} = \hmu_{\MLE}$, but the credible interval is wider, as stated in the Theorem \ref{thm:param}.

Because the shape parameter, $\vtheta$, enters the definition of the covariance kernel in a non-trivial way, the only way to treat it as a hyperparameter and assign a tractable prior would be for the prior to be discrete.  We believe that choosing such a prior in practice involves too much guesswork, so we choose to use either $\vtheta_\MLE$ or $\vtheta_\GCV$.


%%%%%%%%%%%%%%%%%%%%%%%%%%%%%%%%%%%%%%%%%%%%%%%%%%%%%%%%%
%%%%%%%%%%%%%%%%%%%%%%%%%%%%%%%%%%%%%%%%%%%%%%%%%%%%%%%%%
\subsubsection{Proof for Generalized Cross-Validation} \label{sec:GCV}
%%%%%%%%%%%%%%%%%%%%%%%%%%%%%%%%%%%%%%%%%%%%%%%%%%%%%%%%%
%%%%%%%%%%%%%%%%%%%%%%%%%%%%%%%%%%%%%%%%%%%%%%%%%%%%%%%%%
A third parameter optimization technique is \emph{leave-one-out cross-validation} (CV).  Let $\mathring{y}_i = \Ex[f(\vx_i ) | \vf_{-i} = \vy_{-i}]$, where the subscript $-i$ denotes the vector excluding the $i^{\text{th}}$ component.  This is the conditional expectation of $f(\vx_i )$ given the parameters $m$, $s$, and $\vtheta$, and all data but the function value at $\vx_i$.  The cross-validation criterion, which is to be minimized, is the sum of squares of the difference between these conditional expectations and the observed values:
\begin{align} \label{FJH:eq:CVA}
\textup{CV} = \sum_{i=1}^n (y_i - \mathring{y}_i)^2.
\end{align}

Let $\mA = \mC^{-1}_\vtheta$, let $\vzeta = \mA (\vy - m \vone)$, and partition $\mC_\vtheta$, $\mA$, and $\vzeta$ as
\begin{gather*}
\mC_\vtheta = \begin{pmatrix} c_{ii}  & \vC_{-i,i}^T \\  \vC_{-i,i} & \mC_{-i,-i}\end{pmatrix}, \qquad
\mA = \begin{pmatrix} a_{ii}  & \vA_{-i,i}^T \\  \vA_{-i,i} & \mA_{-i,-i}\end{pmatrix}, \\ \vzeta = \begin{pmatrix} \zeta_i   \\  \vzeta_{-i} \end{pmatrix},
\end{gather*}
where the subscript $i$ denotes the $i^{\text{th}}$ row or column, and the subscript $-i$ denotes all rows or columns except the $i^{\text{th}}$. Following this notation, Lemma \ref{thrm:condDist} implies that 
\begin{align*}
\mathring{y}_i & = m + \vC^T_{-i,i} \mC_{-i,-i}^{-1} (\vy_{-i} -m \vone)  \\
\zeta_i  & = a_{ii}(y_i - m) + \vA_{-i,i}^T(\vy_{-i} - m \vone) \\
& = a_{ii}[(y_i - m) - \vC^T_{-i,i} \mC_{-i,-i}^{-1} (\vy_{-i} -m \vone)] \\
& = a_{ii}(y_i - \mathring{y}_i).
\end{align*}
Thus, \eqref{FJH:eq:CVA} may be re-written as 
\begin{align*} %\label{FJH:eq:CVB}
\textup{CV} = \sum_{i=1}^n \left(\frac{\zeta_i}{a_{ii}} \right)^2, \qquad \vzeta = \mC^{-1}_\vtheta(\vy - m \vone).
\end{align*}
The \emph{generalized cross-validation} criterion (GCV) replaces the $i^{\text{th}}$ diagonal element of $\mA$ in the denominator by the average diagonal element of $\mA$ \cite{CraWah79a,GolHeaWah79a,Wah90}:
\begin{align*} 
\textup{GCV} &
= \frac{\sum_{i=1}^n\zeta_i^2}{\left(\frac 1n \sum_{i=1}^n a_{ii} \right)^2} \\
& = \frac{(\vy - m\vone)^T \mC^{-2}_\vtheta (\vy - m \vone)}{\left(\frac 1n \trace(\mC_\vtheta^{-1}) \right)^2}.
\end{align*}

The loss function $\textup{GCV}$ depends on $m$ and $\vtheta$, but not on $s$.  Minimizing the GCV  yields the formulae in Theorem \ref{thm:param} for $\widehat{\mu}_{\textup{GCV}}$ and $\vtheta_{\textup{GCV}}$.  
Plugging the value of $m_\GCV$ into \eqref{eqn:BayesCub} yields the formulae in Theorem \ref{thm:param} for $\widehat{\mu}_{\textup{GCV}}$.

An estimate for $s$ may be obtained by noting that by Lemma \ref{thrm:condDist},
\begin{align*}
\var[f(\vx_i ) | \vf_{-i} = \vy_{-i}] = s^2 a_{ii}^{-1}.
\end{align*}
Thus, we may estimate $s$ using an argument similar to that used in deriving the GCV and then substituting $m_{\textup{GCV}}$ for $m$:
\begin{align*}
s^2 &= \var[f(\vx_i ) | \vf_{-i} = \vy_{-i}] a_{ii} \\ 
& \approx \frac 1n \sum_{i=1}^n (y_i - \mathring{y}_i)^2a_{ii}
 = \frac 1n \sum_{i=1}^n \frac{\zeta_i^2}{a_{ii}} \\ 
 & \approx \frac{ \frac 1n \sum_{i=1}^n \zeta_i^2}{\frac 1n \sum_{i=1}^n a_{ii} } = \frac{(\vy - m\vone)^T \mC_\vtheta^{-2} (\vy - m \vone)}{ \trace(\mC_\vtheta^{-1}) } \\ 
 & \approx  s^2_{\textup{GCV}}.
\end{align*}
where $s^2_{\textup{GCV}}$ is given by the formula in Theorem \ref{thm:param}.

The credible interval based on GCV corresponds to \eqref{eqn_prob_confidence_interval} with the estimated $m$, $s$, and $\vtheta$.  This completes the proof of Theorem \ref{thm:param}.


%%%%%%%%%%%%%%%%%%%%%%%%%%%%%%%%%%%%%%%%%%%%%%%%%%%%%%%%%
%%%%%%%%%%%%%%%%%%%%%%%%%%%%%%%%%%%%%%%%%%%%%%%%%%%%%%%%%
\subsection{The automatic Bayesian cubature algorithm}
\label{sec:bayes_cubature_algo}
%%%%%%%%%%%%%%%%%%%%%%%%%%%%%%%%%%%%%%%%%%%%%%%%%%%%%%%%%
%%%%%%%%%%%%%%%%%%%%%%%%%%%%%%%%%%%%%%%%%%%%%%%%%%%%%%%%%
The previous section presents three credible intervals, \eqref{eqn_prob_CI}, for $\mu$, the desired integral.  Each credible interval is based on different assumptions about the hyperparameters $m$, $s$, and $\vtheta$.  We stress that one must estimate these hyperparameters or assume a prior distribution on them because the credible intervals are used as stopping criteria for our cubature rule.  Since a credible interval makes a statement about a typical function---not an outlier---one must try to ensure that the integrand is a typical draw from the assumed Gaussian stochastic process.

Our  Bayesian cubature algorithm increases the sample size until the width of the credible interval is small enough.  This is accomplished through successively doubling the sample size.  The steps are detailed in Algorithm \ref{algorithm1}.

\algnewcommand{\IIf}[1]{\State\algorithmicif\ #1\ \algorithmicthen\ }
\algnewcommand{\IElse}{\unskip\ \algorithmicelse\ }
\algnewcommand{\EndIIf}{\unskip\ \algorithmicend\ \algorithmicif}


\begin{algorithm}
\caption{Automatic Bayesian Cubature}\label{algorithm1}
  \begin{algorithmic}[1]
  	\Require a generator for the sequence
  	$\vx_1, \vx_2, \ldots$; 
  	a black-box function, $f$; 
  	an absolute error tolerance,
  	$\varepsilon>0$; the positive initial sample size, $n_0$;
  	the maximum sample size $n_{\textup{max}}$
  	
      \State $n \gets n_0, \; n' \gets 0, \; \err_{\textup{CI}} \gets \infty$
      
      \While{$\err_{\textup{CI}} > \varepsilon$ and $n \le n_{\textup{max}}$}
      
        \State\label{LoopStart}Generate $\{ \vx_i\}_{i=n' + 1}^{n}$ and sample $\{f(\vx_i)\}_{i=n'+1}^{n}$
        \State Compute $\vtheta$ by \eqref{eqn:thetaMLE} or \eqref{vthetaGCV}
        \State Compute $\err_{\textup{CI}}$  according to \eqref{eqn:errMLE}, \eqref{FJH:eq:errFull}, or \eqref{GCVerr}
        
       	\State	$n' \gets n, \; n \gets 2n'$
        
        \EndWhile
        
        \State Update sample size to compute $\hmu$, $n \gets n'$
        \State Compute $\hmu$, the approximate integral,   according to \eqref{eqn:cubMLE} or \eqref{eqn:muCV}
      \State \Return $\hmu, \; n$  and $\err_{\textup{CI}}$
  \end{algorithmic}
\end{algorithm}

\FJHNote{Add something about the cost.}

%%%%%%%%%%%%%%%%%%%%%%%%%%%%%%%%%%%%%%%%%%%%%%%%%%%%%%%%%
%%%%%%%%%%%%%%%%%%%%%%%%%%%%%%%%%%%%%%%%%%%%%%%%%%%%%%%%%
\subsection{Example with the Mat\'ern kernel} \label{MVN_example}
%%%%%%%%%%%%%%%%%%%%%%%%%%%%%%%%%%%%%%%%%%%%%%%%%%%%%%%%%
%%%%%%%%%%%%%%%%%%%%%%%%%%%%%%%%%%%%%%%%%%%%%%%%%%%%%%%%%

To demonstrate automatic Bayesian cubature consider a Mat\'ern covariance kernel:
\begin{align*}
C_{\theta}(\vx, \vt) = \prod_{k=1}^d \exp(-\theta|\vx_k-\vt_k|)(1+\theta |\vx_k-\vt_k|),
\end{align*}
and Sobol' points as the nodes.  (Sobol' points are a typical space-filling design.)
Also, consider the integration problem of evaluating  \emph{multivariate normal probabilities}:
\begin{align}
\label{eqn:GaussDef}
\mu = \int_{(\va,\vb)} \frac{\exp\bigl(- \frac 12 \vt^T \mSigma^{-1} \vt \bigr)}{\sqrt{(2 \pi)^{d'} \det(\mSigma)}} \, \dvt,
\end{align}
where $(\va,\vb)$ is a finite, semi-infinite or infinite box in $\reals^{d'}$.  This integral does not have an analytic expression for general $\mSigma$, so cubatures are required.  

Genz \cite{Gen93} introduced a variable transformation to transform \eqref{eqn:GaussDef} into an integral on the unit cube.  Let $\mSigma= \mL \mL^T$ be the Cholesky decomposition where $\mL = (l_{jk})_{j,k=1}^d$ is a lower triangular matrix.  Iteratively define
\begin{align*}
\alpha_1& = \Phi(a_1), \qquad \beta_1 = \Phi(b_1)
\\
\alpha_j&(x_1,...,x_{j-1}) = 
\\
&\; \Phi
\left(
\frac{1}{l_{jj}} 
\left(
a_j - \sum_{k=1}^{j-1} l_{jk} \Phi^{-1}(\alpha_k + x_k(\beta_k-\alpha_k))
\right)
\right), 
\\
&\hspace{5cm} j=2,...,d,
\\
\beta_j&(x_1,...,x_{j-1}) = 
\\
&\; \Phi
\left(
\frac{1}{l_{jj}} 
\left(
b_j - \sum_{k=1}^{j-1} l_{jk} \Phi^{-1}(\alpha_k + x_k(\beta_k-\alpha_k))
\right)
\right), 
\\
&\hspace{5cm} j=2,...,d,
\end{align*}
\begin{align}
\label{fGenzdef}
f_{\text{Genz}}(\vx) = \prod_{j=1}^d [\beta_j(\vx) - \alpha_j(\vx)].
\end{align}
where $\Phi$ is the cumulative standard normal distribution function.  Then, $\mu = \int_{[0,1]^{d'-1}} f_{\text{Genz}}(\vx) \, \dvx$. As we can see, this approach transforms $d'$ integral into a $d=d'-1$ integral.

\begin{figure}
	\captionsetup[subfigure]{labelformat=empty}
	%\begin{subfigure}[h]{0.48\linewidth}
	%	\includegraphics[width=1.1\linewidth]{Plotting_gaussian}
	%\end{subfigure}
	\centering
	%\begin{subfigure}[h]{0.48\linewidth}
		\includegraphics[width=0.7\linewidth]{GenzFunc_varTx_none}
	%\end{subfigure}
	\caption{The $d'=3$ multivariate normal probability transformed to an integral of $f_{\text{Genz}}$ of $d=2$. This plot can be reproduced using \code{IntegrandPlots.m} in GAIL}
	\label{fig:MVN_Genz}
\end{figure}

We use the following parameter values in the simulation: 
\begin{align*}
d' = 3, \quad \va = \begin{pmatrix}
-6 \\ -2 \\ -2
\end{pmatrix}, \quad 
\vb = \begin{pmatrix}
5 \\ 2 \\ 1
\end{pmatrix} , \quad 
\mL = \begin{pmatrix}
4 & 1 & 1 \\ 0 & 1 & 0.5 \\ 0 & 0 & 0.25
\end{pmatrix}.
\end{align*}
The node sets are randomly scrambled Sobol' points \cite{DicEtal14a,DicPil10a}.  The results are for 400 randomly chosen $\varepsilon$ in the interval $[10^{-5}, 10^{-2}]$ as shown in \figref{fig:MVN_Metern_d2b2}. In each run, the nodes are randomly scrambled.  We  observe the algorithm meets the error criterion 95\% of the time.
%with Intel i7 3630QM and 16GB RAM memory
%On our test computer, 
As shown in \figref{fig:MVN_Metern_d2b2}, computation time increases rapidly with $n$. 
The  maximum likelihood estimation of $\vtheta$, which requires repeated evaluation of the objective function, is the most time consuming of all. It takes tens of seconds to compute $\hmu_n$ once with $\varepsilon = 10^{-5}$.   In contrast, this example in Section \ref{sec:NumExp} take less than a hundredth of a second to compute $\hmu_n$ once with $\varepsilon = 10^{-5}$ using our new algorithm. Not only is the Bayesian cubature with the Mat\'ern kernel slow, but also $\mC_\vtheta$ becomes highly ill-conditioned as $n$ increases.
So, Algorithm \ref{algorithm1}  in its current form is impractical. 
%\FJHNote{Jags, please check this re-write} %\JRNote{looks good}

\begin{figure}
	\centering
		\includegraphics[width=0.95\linewidth]{MVN_guaranteed_time_Matern_d2_2019-Jun-29}
	\centering
		\includegraphics[width=0.95\linewidth]{MVN_rapid_n_vs_time_Matern_d2_2019-Jun-29}
	\caption{Multivariate normal probability estimated using Mat\'ern kernel with $d=2$ using empirical stopping criterion. Top: Guaranteed integration within error tolerance $\varepsilon$. Bottom: Computation time rapidly increases with increase of $n$. These figures can be reproduced using \code{matern\_guaranteed\_plots.m} in GAIL.}
	\label{fig:MVN_Metern_d2b2}
\end{figure}

%%%%%%%%%%%%%%%%%%%%%%%%%%%%%%%%%%%%%%%%%%%%%%%%%%%%%%%%%
%%%%%%%%%%%%%%%%%%%%%%%%%%%%%%%%%%%%%%%%%%%%%%%%%%%%%%%%%
\section{Fast Automatic Bayesian Cubature}\label{sec:fast_BC}
%%%%%%%%%%%%%%%%%%%%%%%%%%%%%%%%%%%%%%%%%%%%%%%%%%%%%%%%%
%%%%%%%%%%%%%%%%%%%%%%%%%%%%%%%%%%%%%%%%%%%%%%%%%%%%%%%%%

The generic automatic Bayesian cubature algorithm described in the previous section requires $\Order(n^3)$ operations to estimate $\vtheta$, compute the credible interval width, and compute the cubature. Now we explain how to speed up the calculations. A key is to choose covariance kernels that match the nodes, $\{\vx_i\}_{i=1}^n$, so that the vector-matrix operations required by Bayesian cubature can be accomplished using fast transforms at a cost of $\Order(n \log(n))$.

%%%%%%%%%%%%%%%%%%%%%%%%%%%%%%%%%%%%%%%%%%%%%%%%%%%%%%%%%
%%%%%%%%%%%%%%%%%%%%%%%%%%%%%%%%%%%%%%%%%%%%%%%%%%%%%%%%%
\subsection{Fast Transform Kernel}
%%%%%%%%%%%%%%%%%%%%%%%%%%%%%%%%%%%%%%%%%%%%%%%%%%%%%%%%%
%%%%%%%%%%%%%%%%%%%%%%%%%%%%%%%%%%%%%%%%%%%%%%%%%%%%%%%%%
We make some assumptions about the relationship between the covariance kernel and the nodes, which will be shown to hold in Section \ref{sec:shift_invariant_kernel} for rank-1 lattices and shift-invariant kernels.  Although the integrands and covariance kernels are real, it is convenient to allow related vectors and matrices to be complex.  A relevant example is the fast Fourier transform (FFT) of a real-valued vector, which is a complex-valued vector. 

We introduce some notation:
\begin{align}
\nonumber
\mC = \mC_\vtheta &= \Big(C_\vtheta(\vx_i,\vx_j)\Big)_{i,j=1}^n  = (\vC_1,...,\vC_n) 
\\
\label{eqn:ftk_factor}
&= \frac 1n \mV \mLambda \mV^H , 
\quad \quad \mV^H = n \mV^{-1}, \\
\nonumber
\mV &= (\vv_1,...,\vv_n)^T = (\vV_1,...,\vV_n), \\
\nonumber
\mC^p  &= \frac 1n \mV \mLambda^{p} \mV^H, \qquad \forall p \in \integers.
\end{align}
Here the $\vC_j$ are the columns of $\mC$.  In this and later sections, we drop the $\vtheta$ dependence of various quantities for simplicty of notation.  Here, $\mV^H$ is the Hermitian of $\mV$, $\vC_1,...,\vC_n$ are columns of $\mC$.  $\vV_1,...,\vV_n$ are columns of $\mV$ and $\vv_1,...,\vv_n$ are rows of $\mV$.  The normalization of $\mV$ assumed in \eqref{eqn:ftk_factor} conveniently allows the first eigenvector, $\vV_1$, to be the vector of ones in \eqref{fastcompAssumpB} below.  The columns of matrix $\mV$ are eigenvectors of $\mC$, and $\mLambda$ is a diagonal matrix of eigenvalues of $\mC$.
For any $n \times n$ vector $\vb$, define the notation  $\widetilde{\vb} := \mV^H \vb$.

We make three assumptions that allow the fast computation:
\begin{subequations} \label{fastcompAssump}
	\begin{gather}
	\label{fastcompAssumpA}
	\mV \text{ may be identified analytically}, \\
	\label{fastcompAssumpB}
	\vv_1 = \vV_1 = \vone, \\
	\label{fastcompAssumpC}
	\mV^H \vb  \text{ requires only $\Order(n \log(n))$ operations } \forall \vb.
	\end{gather}
\end{subequations}
We call the transformation $\vb \mapsto \mV^H \vb$ a \emph{fast transform} and $C$ a \emph{fast transform kernel}.  

Under assumptions \eqref{fastcompAssump} the eigenvalues may be identified as the fast transform of the first column of $\mC$:
\begin{align}
\nonumber
\vlambda 
& = \begin{pmatrix}
\lambda_1 \\ \vdots \\ \lambda_n
\end{pmatrix} = \mLambda \vone = \mLambda \vv_1^* 
= \underbrace{\left( \frac 1n \mV^H  \mV \right) }_{\mathsf{I}} \mLambda \vv_1^* \\
&= \mV^H \left( \frac 1n \mV \mLambda \vv_1^* \right)
= \mV^H \vC_1 =  \widetilde{\vC}_1,
\label{eqn:fast_transform_to_eigvalues}
\end{align}
Where $\mathsf{I}$ is the identity matrix.
Also note that the fast transform of $\vone$ has a simple form
\begin{align*} 
\widetilde{\vone}
& = \mV^H \vone = \mV^H \vV_1 = 
\left(n, 0, \dots, 0 \right)^T.
% \begin{pmatrix}n  \\ 0 \\ \vdots \\ 0 \end{pmatrix}.
\label{eqn:fast_transform_one}
\end{align*}

Many of the terms that arise in the calculations in  Algorithm \ref{algorithm1} take the form $\va^T\mC^{p}\vb$ for real $\va$ and $\vb$ and integer $p$.  These can be calculated via the transforms $\widetilde{\va} = \mV^H \va$ and $\widetilde{\vb} = \mV^H \vb$ as 
\begin{equation*}
\va^T\mC^p\vb = \frac 1n \va^T \mV \mLambda^p \mV^H \vb
= \frac 1n \widetilde{\va}^H\mLambda^p \widetilde{\vb}
= \frac 1n \sum_{i=1}^n \lambda_i^p \widetilde{a}_i^* \widetilde{b}_i.
\end{equation*}
Note that $\widetilde{\va}^*$ appears on the right side of this equation because $\va^T \mV = (\mV^H \va)^* = \widetilde{\va}^*$.  In particular,
\begin{align*}
\vone^T\mC^{-p}\vone & = \frac{n}{\lambda_1^p},
&
\vone^T\mC^{-p}\vy &= \frac{\widetilde{y}_1}{\lambda_1^p},
\\
\vy^T\mC^{-p} \vy &= \frac 1n \sum_{i=1}^n \frac{\abs{\widetilde{y}_i}^2}{\lambda_i^p},
&
\vc^T\mCInv \vone &= \frac{\widetilde{c}_1}{\lambda_1},\\
\vc^T\mCInv \vy &= \frac 1n \sum_{i=1}^n \frac{\widetilde{c}_i^* \widetilde{y}_i}{\lambda_i}, & 
\vc^T\mCInv \vc &= \frac 1n \sum_{i=1}^n \frac{\abs{\widetilde{c}_i}^2}{\lambda_i},
\end{align*}
where $\widetilde{\vy} = \mV^H \vy$ and 
$\widetilde{\vc} = \mV^H \vc$.  For any real $\vb$, with $\widetilde{\vb} = \mV^H\vb$, it follows that $\widetilde{b}_1$ is real since the first row of $\mV^H$ is $\vone$.

The covariance kernel used in practice also may satisfy an additional assumption:
\begin{align} \label{addAssump}
\int_{[0,1]^d} C_{\vtheta}(\vt,\vx) \, \D \vt = 1 \qquad \forall \vx \in [0,1]^d,
\end{align}
which implies that $c_{0\vtheta} = 1$ and $\vc_\vtheta = \vone$.  Under \eqref{addAssump}, the expressions above may be further simplified:
\begin{align*}
\vc^T\mCInv \vone =
\vc^T\mCInv \vc = \frac{n}{\lambda_1}.
\end{align*}
The assumptions and derivations above lead to the following theorem.

\begin{theorem} \label{thm:fastparam}
Under assumptions \eqref{fastcompAssump}, the parameters and credible interval widths in Theorem \ref{thm:param} may be expressed in terms of the fast transforms of the function data, the first column of the Gram matrix, and $\vc_\vtheta$ as follows:
\allowdisplaybreaks
\begin{align}
\nonumber
m_\MLE &=  m_{\full} = m_{\GCV} =  \frac{\widetilde{y}_1}{n} = \frac 1n \sum_{i=1}^n y_i,
\\
\nonumber
s^2_\MLE 
& =
\frac{1}{n^2} 
\sum_{i=2}^n \frac{\abs{\widetilde{y}_i}^2}{\lambda_i}, \\
\nonumber
\widehat{\sigma}^2_{\textup{full}}
& =
\frac{1}{n(n-1)} \sum_{i=2}^n \frac{\abs{\widetilde{y}_i}^2}{\lambda_i}
\\
\nonumber
& \qquad \times
\left[\frac{\lambda_1}{n}{\left(1 - \frac{\widetilde{c}_1}{\lambda_1}\right)^2} + \left(c_0  - \frac 1n \sum_{i=1}^n \frac{\abs{\widetilde{c}_i}^2}{\lambda_i}\right) \right], \\
\nonumber 
s^2_{\textup{GCV}} & : =  \frac 1{n} \sum_{i=2}^n \frac{\abs{\widetilde{y}_i}^2}{\lambda_i^2}  \left [ \sum_{i=1}^n \frac{1}{\lambda_i} \right]^{-1},\\
\label{eqn_MLE_loss_func_optimized_2} %\nonumber
\vtheta_\MLE
&= 
\argmin_{\vtheta}
\left[
\log\left(
\sum_{i=2}^n \frac{\abs{\widetilde{y}_i}^2}{\lambda_{\vtheta i}}
\right) %\right .\\ \hspace{3cm} \left .
 + \frac{1}{n}\sum_{i=1}^n \log(\lambda_{\vtheta i})
\right],\\
\nonumber 
\vtheta_{\GCV} 
&= \argmin_\vtheta \left[ \log \left ( \sum_{i=2}^n \frac{\abs{\widetilde{y}_i}^2}{\lambda_{\vtheta i}^2} 
\right) \right .\\
\label{thetaGCV}
& \qquad \qquad \qquad \qquad  \left . -2\log\left( \sum_{i=1}^n \frac{1}{\lambda_{\vtheta i}} \right)
\right], \\
\nonumber
\hmu_\MLE  &= \hmu_{\full} = \hmu_{\GCV} =
\frac{\widetilde{y}_1}{n} +
\frac 1n \sum_{i=2}^n \frac{ \widetilde{c}_i^* \widetilde{y}_i}{\lambda_i}, \\
\nonumber
\err_\MLE  &
=
\frac{2.58}{n}\sqrt{
	\sum_{i=2}^{n} \frac{\abs{\widetilde{y}_i}^2}{\lambda_i}  
	\,
	\left( c_0 - \frac 1n \sum_{i=1}^n \frac{\abs{\widetilde{c}_i}^2}{\lambda_i} \right) 
}, \\
\nonumber
\err_{\full} & = t_{n-1,0.995} \hsigma_{\textup{full}}, \\
\nonumber
\err_{\textup{GCV}} & =
\frac{2.58}{n}\left\{\sum_{i=2}^n \frac{\abs{\widetilde{y}_i}^2}{\lambda_i^2}  \left [ \frac 1n \sum_{i=1}^n \frac{1}{\lambda_i} \right]^{-1} \right.
\\ 
\nonumber
&\qquad \qquad \left . \times
\left( c_0 - \frac 1n \sum_{i=1}^n \frac{\abs{\widetilde{c}_i}^2}{\lambda_i} \right) 
\right\}^{1/2}.
\end{align}
Under the further assumption \eqref{addAssump}, it follows that 
\begin{align}
\nonumber
\hmu_\MLE  &= \hmu_{\full} = \hmu_{\GCV} =
\frac{\widetilde{y}_1}{n} = \frac 1n \sum_{i=1}^n y_i,
\end{align}
and so $\hmu$ is simply the sample mean.  Also, under assumption \eqref{addAssump}, the credible interval widths simplify to
\begin{align}
\label{eq:errMLEAllAsump}
\err_\MLE  &
=
\frac{2.58}{n}\sqrt{
	\sum_{i=2}^{n} \frac{\abs{\widetilde{y}_i}^2}{\lambda_i}  
	\,
	\left( 1 -  \frac{n}{\lambda_1} \right) 
}, \\
\label{FJH:eq:errFullSimple}
\err_{\textup{full}}
&=
t_{n-1,0.995}
\sqrt{\frac{1}{n(n-1)} \sum_{i=2}^n \frac{\abs{\widetilde{y}_i}^2}{\lambda_i}  \left(\frac{\lambda_1}{n}  - 1  \right)}, \\
\nonumber
\err_{\textup{GCV}} & =
\frac{2.58}{n}\left\{\sum_{i=2}^n \frac{\abs{\widetilde{y}_i}^2}{\lambda_i^2}  \left [ \frac 1n \sum_{i=1}^n \frac{1}{\lambda_i} \right]^{-1} 
\right. \\ &\qquad \qquad \left . \times
\left( 1 -  \frac{n}{\lambda_1} \right)  
\right\}^{1/2}. \label{errGCVSimple}
\end{align}
\end{theorem}




\section{Integration Lattices and Shift Invariant Kernels}
\label{sec:shift_invariant_kernel}

The preceding sections lay out an automatic Bayesian cubature algorithm whose computational cost is only $\Order(n \log(n))$ if $n$ function values are used.  However, this algorithm relies on covariance kernel functions, $C$ and node sets, $\{\vx_i\}_{i=1}^n$ that satisfy assumptions \eqref{fastcompAssump}.  We  also want to satisfy assumption \eqref{addAssump}.  
%To facilitate the fast transform, it is assumed in this section and the next that $n$ is power of $2$.  
To facilitate the fast transform, $n$ must be power of $2$.  

\subsection{Extensible Integration Lattice Node Sets}

The set of nodes used is defined by a shifted extensible integration lattice node sequence, which takes the form
\begin{equation*}
\vx_{i} = \vh \phi(i-1) + \vDelta \mod \vone, \qquad i \in \naturals.
\end{equation*} 
Here, $\vh$ is a $d$-dimensional generating vector of positive integers, $\vDelta$ is some point in $[0,1)^d$, often chosen at random, and $\{\phi(i)\}_{i=0}^n$ is the van der Corput sequence, defined by reflecting the binary digits of the integer about the decimal point, i.e., 
\begin{align} \label{vdCDef}
\begin{array}{r|ccccccccccccc}
i & 0 & 1 & 2 & 3 & 4 &  5 & 6 & 7 & \cdots \\
i & 0_2 & 1_2 & 10_2 & 11_2 & 100_2 & 101_2 & 110_2 & 111_2  & \cdots\\
\toprule
\phi(i) & {}_2.0 &  {}_2.1 & {}_2.01 &  {}_2.11  & {}_2.001 &  {}_2.101 & {}_2.011 &  {}_2.111 & \cdots\\
\phi(i) & 0 &  0.5 &  0.25 & 0.75 &  0.125 & 0.625  &  0.375 & 0.875 & \cdots
\end{array}
\end{align}

An example of $64$ nodes is given in \figref{latticefig}.  The even coverage of the unit cube is ensured by a well chosen generating vector.  The choice of generating vector is typically done offline by computer search.  See \cite{DicEtal14a,HicNie03a} for more on extensible integration lattices.
\begin{figure}[htp]
	\centering
	\includegraphics[height=5cm]{ShiftedLatticePoints}
	\caption{Example of a shifted integration lattice node set  in $d=2$. 
	This figure can be reproduced using \code{PlotPoints.m} in GAIL} \label{latticefig}
\end{figure}

\subsection{Shift Invariant Kernels}
The covariance functions $C$ that match integration lattice node sets have the form
\begin{align} \label{eq:shInv}
C(\vt,\vx) = K(\vt - \vx \bmod \vone).
\end{align}
This is called a \emph{shift invariant kernel} because shifting both arguments of the covariance function by the same amount leaves the value unchanged.   By a proper scaling of the kernel $K$ it follows that assumption \eqref{addAssump} is satisfied. Of course, $K$ must also be of the form that ensures that $C$ is symmetric and positive definite, as assumed in \eqref{FJH:eq:CondPosDef}. 

A family of shift invariant kernels is constructed via even degree Bernoulli polynomials:
\begin{multline}
\label{the_kernel_eqn_bernoulli}
C_\vtheta(\vt, \vx) =
\prod_{l=1}^d \biggl[
1 - (-1)^{r} \eta B_{2r}( |{x_l-t_l}| ) \biggr], \\  
\forall \vt,\vx \in [0,1]^d, \  \vtheta = (r,\eta), \ r \in \naturals, \ \eta > 0.
\end{multline}
Symmetric, periodic, positive definite kernels of this form appear in  \cite{DicEtal14a,Hic96a}.  Bernoulli polynomials are described in \cite[Chapter 24]{OlvEtal10a}.

Larger $r$ implies a greater degree of smoothness of the kernel.  Larger $\eta$ implies greater fluctuations of the output with respect to the input.  Plots of $C(\cdot, 0.3)$ are given in \figref{fig:fourierkernel-dim1} for various $r$ and $\eta$ values.

\begin{figure}
	\centering
	\includegraphics[width=0.9\linewidth]{"fourier_kernel_dim_1"}
	\caption[Fourier kernel]{Shift invariant kernel in 1D shifted by 0.3 to show the discontinuity. This figure can be reproduced using \code{plot\_fourier\_kernel.m} in GAIL. }
	\label{fig:fourierkernel-dim1}
\end{figure}

\subsection{Eigenvectors}
For general shift-invariant covariance functions, the Gram matrix takes the form
\begin{align*}
%\label{shInvKernGramMatrix}
\mC &= \bigl ( C(\vx_i, \vx_j) \bigr)_{i, j = 1}^n \\
& = \Bigl ( K(\vh(\phi(i-1) - \phi(j-1)) \bmod \vone ) \Bigr)_{i, j = 1}^n.
\end{align*}
Here we show the resulting matrix $\mC$ is circulant.
Then using the well known fact, circulant matrices are diagonalized by discrete Fourier transform (DFT), one can obtain the eigenvectors analytically and the eigenvalues using DFT.
% The speedup of their algorithm is in the linear algebra part of the calculations. 
% But the derivation in Sections 4.3 and 4.4 can be obtained much faster by realising that their definition of the covariance matrix,
By the definition of $\mC$, $i,j$th element
\begin{align*}
\mC_{i,j} & = K( x_i - x_j \bmod 1 ) \\
&= K\bigl( \bigl(\vh \phi(i-1) - \vh \phi(j-1)\bigr) \bmod 1 \bigr) 
\end{align*}
Noting that the sequence $\{\phi(i-1)\}_{i=1}^n$ is a re-ordering of $0, \ldots, 1-1/n$ for $n$ a power of $2$, and $\vh$ is a vector integers, one can rewrite
\begin{align*}
\mC_{i,j} &= K_{\vh}\bigl( \bigl(\pi_{2^m}(i-1) - \pi_{2^m}(j-1)\bigr) \bmod 2^m \bigr)
\end{align*}
where $\pi_{2^m}(i-1) = 2^m\phi(i-1)$ is the ''integer version'' of the radical inverse function for the full initial sequence of $2^m$ points and $K_{\vh}(\vt) = K(\vh^T \vt)$. By similar approach
\begin{align*}
\mC_{i,j} &= K_{\vh,\pi_{2^m}}( (i-1) - (j-1) \bmod 2^m ) , 
\\
& \hspace{4cm} \text{for}\; i,j=1,...,2^m ,
\end{align*}
is a permutation of a circulant matrix.
 
The function $\pi_{2^m}$ is a permutation on $\{0, 1, ..., 2^m-1\}$, undoing this
permutation on the rows and columns of the matrix $\mC$ one obtains a circulant
matrix:
$P_{2^m}^T \mC P_{2^m}$.
%The current Sections 4.3 and 4.4 are therefore not needed. See further.
Eigenvectors of the circulant matrices are 
\begin{align} \label{latticeVdef}
\mV = \Bigl ( \me^{2 \pi n \sqrt{-1} \phi(i-1)\phi(j-1)} \Bigr)_{i,j = 1}^n.
\end{align}
and the eigenvalues are
\begin{align*}
\vlambda = \mV \vC_1
\end{align*}
which automatically follow from the properties of circulant matrices.
This proves the assumption $\eqref{fastcompAssumpB}$.


\iftrue
\JRNote{Original content of Section 4.3 begins from here:}


We now demonstrate that the eigenvector matrix for $\mC$ is 
\begin{align} \label{latticeVdef}
\mV = \Bigl ( \me^{2 \pi n \sqrt{-1} \phi(i-1)\phi(j-1)} \Bigr)_{i,j = 1}^n.
\end{align}
Assumption \eqref{fastcompAssumpB} follows automatically.
Now, note that the $k,j$ element of $\mV^H\mV$ is
\begin{equation*}
\sum_{i=1}^n \me^{2 \pi n \sqrt{-1} \phi(i-1)[\phi(j-1) - \phi(k-1)]} .
\end{equation*}
Noting that the sequence $\{\phi(i-1)\}_{i=1}^n$ is a re-ordering of $0, \ldots, 1-1/n$ for $n$ a power of $2$, this sum may be re-written by replacing $\phi(i-1)$ by $(i-1)/n$:
\begin{equation*}
\sum_{i=1}^n \me^{2 \pi \sqrt{-1} (i-1)[\phi(j-1) - \phi(k-1)]}.
\end{equation*}
Since $\phi(j-1) - \phi(k-1)$ is some integer multiple of $1/n$, it follows that this sum is $n \delta_{j,k}$, where $\delta$ is the Kronecker delta function.  This establishes that $\mV^H = n \mV^{-1}$ as in \eqref{eqn:ftk_factor}.

Next, let $\omega_{k, \ell}$ denote the $k,\ell$ element of $\mV^H \mC \mV$, which is given by the double sum
\begin{multline*}
\omega_{k, \ell} = \sum_{i,j=1}^n K(\vh(\phi(i-1) - \phi(j-1)) \bmod \vone ) \\
\times   \me^{-2 \pi n \sqrt{-1} \phi(k-1)\phi(i-1)}  \me^{2 \pi n \sqrt{-1} \phi(j-1)\phi(l-1)}
\end{multline*}
Noting that the sequence $\{\phi(i-1)\}_{i=1}^n$ is a re-ordering of $0, \ldots, 1-1/n$ for $n$ a power of $2$, this sum may be re-written by replacing $\phi(i-1)$ by $(i-1)/n$ and $\phi(j-1)$ by $(j-1)/n$:
\begin{multline*}
\omega_{k, \ell} = \sum_{i,j=1}^n K\left (\vh \left(\frac{i-j}{n} \right) \bmod \vone \right) \\
\times   \me^{-2 \pi \sqrt{-1} \phi(k-1)(i-1)}  \me^{2 \pi \sqrt{-1} (j-1)\phi(\ell-1)}.
\end{multline*}
This sum also remains unchanged if $i$ is replaced by $i+m$ and $j$ is replaced by $j+m$ for any integer $m$:
\begin{multline*}
\omega_{k, \ell} = \sum_{i,j=1}^n K\left (\vh \left(\frac{i-j}{n} \right) \bmod \vone \right) \\
\times   \me^{-2 \pi \sqrt{-1} \phi(k-1)(i+m-1)}  \me^{2 \pi \sqrt{-1} (j+m-1)\phi(\ell-1)} \\
=   \omega_{k, \ell}  \me^{2 \pi \sqrt{-1} m(\phi(\ell-1) - \phi(k-1))}.
\end{multline*}
For this last equality to hold for all integers $m$, we must have $k = \ell$ or $\omega_{k,\ell} = 0$.  Thus, 
\begin{align*}
\omega_{k, \ell} &= \delta_{k,\ell} \sum_{i,j=1}^n K\left (\vh \left(\frac{i-j}{n} \right) \bmod \vone \right) \\
& \qquad \qquad \qquad \qquad \times   \me^{-2 \pi \sqrt{-1} (i - j) \phi(k-1)} \\
& = n \delta_{k,\ell}  \sum_{i=1}^n K\left ( \left(\frac{i\vh}{n} \right) \bmod \vone \right)  \me^{-2 \pi \sqrt{-1} i \phi(k-1) }.
\end{align*}
This establishes $\mV^H \mC \mV$ as a diagonal matrix whose diagonal elements are $n$ times the eigenvalues, i.e., $\lambda_k = \omega_{k,k}/n$.  Furthermore, $\mV$ is the matrix of eigenvectors, which satisfies assumption \eqref{fastcompAssumpA}.
\fi

\subsection{Iterative Computation of the Fast Transform}
Assumption \eqref{fastcompAssumpA} is that computing $\mV^H \vb$ requires only $\Order(n \log(n)) $ operations.  Recall that we assume that $n$ is a power of $2$.  This can be accomplished by an iterative algorithm.  Let $\mV^{(n)}$ denote the $n \times n$ matrix $\mV$ defined in  \eqref{latticeVdef}.  One can compute $\mV^{(2n)H}\vb$ quickly for all $\vb \in \reals^{2n}$ assuming that $\mV^{(n)H}\vb$ can be computed quickly for all $\vb \in \reals^n$, using the butterfly structure of the Fast Fourier Transform (FFT). For the brevity, we only provide the results. 

From the definition of the van der Corput sequence in \eqref{vdCDef}, it follows that
\begin{gather} 
\label{vdCProp}
\phi(2i) = \phi(i)/2, \;  \phi(2i+1) = [\phi(i)+1]/2, \ \ \ i \in \natzero\\
\label{vdCPropB}
\phi(i+n) = \phi(i) + 1/(2n), \qquad i = 0, \ldots, n-1,
\\
\label{vdCPropC}
n \phi(i) \in \natzero, \qquad i = 0, \ldots, n-1,
\end{gather}
still assuming that $n$ is an integer power of two.
Let $\widetilde{\vb} = \mV^{(2n)H}\vb$ for some arbitrary $\vb \in \reals^{2n}$, and define
\begin{gather*}
\vb = \begin{pmatrix} b_1 \\ \vdots \\ b_{2n} \end{pmatrix}, \quad 
\vb^{(1)} = \begin{pmatrix} b_1 \\ \vdots \\ b_{n} \end{pmatrix}, \quad 
\vb^{(2)}  = \begin{pmatrix} b_{n+1} \\ \vdots \\ b_{2n} \end{pmatrix}, \\ 
\widetilde{\vb} = \begin{pmatrix} \widetilde{b}_1 \\ \vdots \\ \widetilde{b}_{2n} \end{pmatrix}, \quad 
\widetilde{\vb}^{(1)} = \begin{pmatrix} \widetilde{b}_1 \\ \widetilde{b}_3 \\ \vdots \\ \widetilde{b}_{2n-1} \end{pmatrix}, \quad 
\widetilde{\vb}^{(2)}  = \begin{pmatrix} \widetilde{b}_{2} \\  \widetilde{b}_{4} \\ \vdots \\ \widetilde{b}_{2n} \end{pmatrix}. 
\end{gather*}
It follows from these definitions and the definition of $\mV$ in  \eqref{latticeVdef} that
\begin{align*}
\widetilde{\vb}^{(1)} &= \left( \sum_{j=1}^{2n}  \me^{-4 \pi n \sqrt{-1} \phi(2i-2)\phi(j-1)} b_{j} \right)_{i=1}^n \\
% &= \left( \sum_{j=1}^{2n}  \me^{-2 \pi n \sqrt{-1} \phi(i-1)\phi(j-1)} b_{j} \right)_{i=1}^n \quad \text{by \eqref{vdCProp}}\\
% &= \left( \sum_{j=1}^{n}  \me^{-2 \pi n \sqrt{-1} \phi(i-1)\phi(j-1)} b_{j} \right)_{i=1}^n \quad \\
% &\qquad \qquad +  \left( \sum_{j=1}^{n}  \me^{-2 \pi n \sqrt{-1} \phi(i-1)\phi(n+j-1)} b_{n+j} \right)_{i=1}^n \\
% &= \mV^{(n)H}\vb^{(1)}  +  \biggl(  \me^{-\pi \sqrt{-1} \phi(i-1)}  \\
% & \quad \times \sum_{j=1}^{n}  \me^{-2 \pi n \sqrt{-1} \phi(i-1)\phi(j-1)} b_{n+j} \biggr)_{i=1}^n \quad  \text{by \eqref{vdCPropB}}\\
&= \mV^{(n)H}\vb^{(1)} +  \left(  \me^{-\pi \sqrt{-1} \phi(i-1)} \right)_{i=1}^n \odot \bigl(\mV^{(n)H}\vb^{(2)} \bigr),
\end{align*}
where $\odot$ denotes the Hadamard (term-by-term) product.  By a similar argument, 
\begin{align*}
\widetilde{\vb}^{(2)} &= \left( \sum_{j=1}^{2n}  \me^{-4 \pi n \sqrt{-1} \phi(2i-1)\phi(j-1)} b_{j} \right)_{i=1}^n \\
% &= \left( \sum_{j=1}^{2n}  \me^{-2 \pi n \sqrt{-1} [\phi(i-1)+1]\phi(j-1)} b_{j} \right)_{i=1}^n \quad \text{by \eqref{vdCProp}}\\
% &= \left( \sum_{j=1}^{n}  \me^{-2 \pi n \sqrt{-1} [\phi(i-1)+1]\phi(j-1)} b_{j} \right)_{i=1}^n \quad \\
% &\qquad  +  \left( \sum_{j=1}^{n}  \me^{-2 \pi n \sqrt{-1} [\phi(i-1)+1]\phi(n+j-1)} b_{n+j} \right)_{i=1}^n \\
% &= \mV^{(n)H}\vb^{(1)} +  \left(  \me^{ -\pi \sqrt{-1} [\phi(i-1) + 1]} \right. \\
% & \qquad \qquad \left . \times \sum_{j=1}^{n}  \me^{-2 \pi n \sqrt{-1} \phi(i-1)\phi(j-1)} b_{n+j} \right)_{i=1}^n \\
% & \hspace{4cm}  \text{by \eqref{vdCPropB} and \eqref{vdCPropC}}\\
&= \mV^{(n)H}\vb^{(1)} -  \left(  \me^{-\pi \sqrt{-1} \phi(i-1)} \right)_{i=1}^n \odot \bigl(\mV^{(n)H}\vb^{(2)} \bigr).
\end{align*}

The computational cost to compute $\mV^{(2n)H}\vb$ is then twice the cost of computing $\mV^{(n)H}\vb^{(1)}$ plus $2n$ multiplications plus $2n$ additions.  An inductive argument shows that $\mV^{(n)H}\vb$ requires only $\Order(n \log(n))$ operations.


\subsection{Overcoming Cancellation Error}
For the kernels used in our computation, it may happen that $n/\lambda_1$ is close to $1$.  Thus, the term $1-n/\lambda_1$, which appears in the credible interval widths, $\err_{\MLE}$, $\err_{\textup{full}}$, and $\err_{\textup{GCV}}$, may suffer from cancellation error.  We can avoid this cancellation error by modifying how we compute the Gram matrix and its eigenvalues.

The shift-invariant kernel of our interest \eqref{the_kernel_eqn_bernoulli} has the form $C = 1 + \rC$. 
One could take advantage of it to avoid the subtraction in $1-n/\lambda_1$. 
Define the new function $\rC : = C -1$, and its associated Gram matrix $\rmC = \mC - \vone \vone^T$, where $\rmC$ is simply the original kernel after discarding ``$1$", there is no actual subtraction involved in the computation.  So we are not really shifting the cancellation error to another place.
Note that $\rC$ inherits the shift-invariant properties of $C$.  
Since $\vone$ is the first eigenvector of $\mC$, it follows that the eigenvalues of $\rmC$ are $\rlambda_1 = \lambda_1 - n, \lambda_2, \ldots, \lambda_n$.  Moreover,
\begin{equation*}
1 - \frac{n}{\lambda_1}  = \frac{\lambda_1 - n}{\lambda_1} = \frac{\rlambda_1}{\rlambda_1 +n},
\end{equation*}
where now the right hand side is free of cancellation error.

We show how to compute $\rC$ without introducing round-off error.  The covariance functions that we use are of product form, namely,
\begin{equation*}
C(\vt, \vx) = \prod_{\ell=1}^d \left[1 + \rC_\ell(t_\ell,x_\ell) \right], \qquad  \rC_\ell:[0,1]^2 \to \reals.
\end{equation*}
Direct computation of $\rC (\vt,\vx) = C(\vt,\vx) -1$ introduces cancellation error if the $ \rC_\ell$ are small.  So, we employ the iteration
\begin{align*}
\rC^{(1)} &= \rC_1(t_1,x_1),  \\
\rC^{(\ell)} &  = \rC^{(\ell-1)}[1 + \rC_\ell(t_\ell,x_\ell)] + \rC_\ell(t_\ell,x_\ell), \\
& \hspace{5cm} \ell = 2, \ldots, d, \\
\rC(\vt,\vx)  & = \rC^{(d)}.
\end{align*}
In this way, the Gram matrix $\rmC$, whose $i,j$-element is $\rC(\vx_i,\vx_j)$ can be constructed with minimal round-off error.

Computing the eigenvalues of $\rmC$ via the procedure given in \eqref{eqn:fast_transform_to_eigvalues} yields $\rlambda_1 = \lambda_1 - n, \lambda_2, \ldots, \lambda_n$. The estimates of $\vtheta$ are computed in terms of the eigenvalues of $\rmC$, so \eqref{eqn_MLE_loss_func_optimized_2} and \eqref{thetaGCV} become
\begin{subequations}
\label{thetaSimple}
\begin{align}
\label{thetaSimpleMLE}
\vtheta_\MLE
&= 
\argmin_{\vtheta}
\left[
\log\left(
\sum_{i=2}^n \frac{\abs{\widetilde{y}_i}^2}{\lambda_i}
\right) 
%\label{thetaMLEsimple}
 + 
 \frac{1}{n}\sum_{i=1}^n \log(\lambda_i)
\right], \\
\label{thetaSimpleGCV}
\vtheta_{\GCV} 
&= \argmin_\vtheta \left[ \log \left ( \sum_{i=2}^n \frac{\abs{\widetilde{y}_i}^2}{\lambda_i^2} 
\right)  -2\log\left( \sum_{i=1}^n \frac{1}{\lambda_i} \right)
\right],
\end{align}
\end{subequations}
where $\lambda_1 = n + \rlambda_1$.  The widths of the credible intervals in \eqref{eq:errMLEAllAsump}, \eqref{FJH:eq:errFullSimple}, and   \eqref{errGCVSimple} become
\begin{subequations}
\label{fastStoppingCriterions}
	\begin{align}
\label{fastStoppingCriterionMLE}
\err_\MLE  &
=
\frac{2.58}{n}\sqrt{
	\frac{\rlambda_1}{\lambda_1}
	\sum_{i=2}^{n} \frac{\abs{\widetilde{y}_i}^2}{\lambda_i}  
}, 
\\
\label{fastStoppingCriterionFull}
\err_{\textup{full}} 
& = \frac{t_{n-1,0.995}}{n} \sqrt{
	\frac{\rlambda_1}{n-1} \sum_{i=2}^n \frac{\abs{\widetilde{y}_i}^2}{\lambda_i}
}, \\
\label{fastStoppingCriterionGCV}
\err_{\textup{GCV}} & =
\frac{2.58}{n}\sqrt{	\frac{\rlambda_1}{\lambda_1} \sum_{i=2}^n \frac{\abs{\widetilde{y}_i}^2}{\lambda_i^2}  \left [ \frac 1n \sum_{i=1}^n \frac{1}{\lambda_i} \right]^{-1}} .
	\end{align}
\end{subequations}
Since $\rlambda_1 = \lambda_1 - n$ and $\lambda_1 \sim n$ it follows $\rlambda_1/\lambda_1 \approx \rlambda_1/(n-1)$ and is small for  large $n$.  Moreover, the credible intervals via empirical Bayes and full Bayes are similar, since $t_{n-1,0.995}$ is approximately $2.58$. 
The computational steps for the improved, faster, automatic Bayesian cubature are detailed in Algorithm \ref{algorithm2}.

\begin{algorithm}
	\caption{Fast Automatic Bayesian Cubature}\label{algorithm2}
	\begin{algorithmic}[1]
		\Require a generator for the rank-1 Lattice sequence
		$\vx_1, \vx_2, \ldots$; 
		a shift-invariant periodic kernel, $C$;
		a black-box function, $f$; 
		an absolute error tolerance,
		$\varepsilon>0$; the positive initial sample size, $n_0$;
		the maximum sample size $n_{\textup{max}}$
		
		\State $n \gets n_0, \; n' \gets 0, \; \err_{\textup{CI}} \gets \infty$
		
		\While{$\err_{\textup{CI}} > \varepsilon$ and $n \le n_{\textup{max}}$}
		
		\State\label{LoopStart}Generate $\{ \vx_i\}_{i=n' + 1}^{n}$ and sample $\{f(\vx_i)\}_{i=n'+1}^{n}$
		\State Compute $\vtheta$ by \eqref{thetaSimpleMLE} or \eqref{thetaSimpleGCV}
		\State Compute $\err_{\textup{CI}}$  according to \eqref{fastStoppingCriterionMLE}, \eqref{fastStoppingCriterionFull}, or \eqref{fastStoppingCriterionGCV}
		
		\State	$n' \gets n, \; n \gets 2n'$
		
		\EndWhile
		
		\State Update sample size to compute $\hmu$, $n \gets n'$
		\State Compute $\hmu$, the approximate integral,   according to \eqref{muhatGCV-FB-MLE-Simple}
		\State \Return $\hmu, \; n$  and $\err_{\textup{CI}}$
	\end{algorithmic}
\end{algorithm}

We summarize the results of this section and the previous one as follows:
\begin{prop}
Any symmetric, positive definite, shift-invariant covariance kernel of the form \eqref{eq:shInv} scaled to satisfy \eqref{addAssump}, when matched with rank-1 lattice data-sites, must satisfy assumptions \eqref{fastcompAssump}.  The \emph{fast Fourier transform} (FFT) can be used to expedite the estimates of $\vtheta$ in \eqref{thetaSimple} and the credible interval widths \eqref{fastStoppingCriterions} in $\Order(n \log(n))$ operations. The cubature, $\hmu$, is just the sample mean.
\end{prop}

We have implemented the fast adaptive Bayesian cubature algorithm in MATLAB as part of the Guaranteed Adaptive Integration Library (GAIL) \cite{ChoEtal17b} as \allowbreak \code{cubBayesLattice\_g}. This algorithm uses the kernel defined in  \eqref{the_kernel_eqn_bernoulli} with  $r=1,2$ and the periodizing variable transforms in  \secref{period_var_tx}.  The rank-1 lattice node generator is taken from \cite{Nuy17a} (\code{exod2\_base2\_m20}).

\section{Numerical Experiments} \label{sec:NumExp}

\subsection{Periodizing Variable Transformations}
\label{period_var_tx}
The shift-invariant covariance kernels underlying our Bayesian cubature  assume that the integrand has a degree of periodicity, with the smoothness assumed depending on the smoothness of the kernel.  While integrands arising in practice may be smooth, they might not be periodic.  Variable transformations can be used to ensure periodicity.

Suppose that the original integral has been expressed as 
\begin{equation*}
\mu := \int_{[0,1]^d} g(\vt) \, \dif \vt
\end{equation*}
where $g$ has sufficient smoothness, but lacks periodicity.  The Baker's transform,
\begin{multline} \label{eq:bakerTrans}
\vPsi: \vx \mapsto (\Psi(x_1),  \ldots, \Psi(x_d)), \\ \Psi(x)  =1 - 2 \abs{x - 1/2},
\end{multline}
allows us to write $\mu$ in the form of \eqref{eqn:defn_mu}, where $f(\vx) = g(\vPsi(\vx))$. Also called tent transform, $\Psi'(x)$ is not defined, not suitable when the $f$ is required to have also smooth derivatives. 

A family of variable transforms that also provide some derivatives take the form
\begin{subequations} %\label{eq:varTrans}
\begin{equation*}
\vPsi: \vx \mapsto (\Psi(x_1),  \ldots, \Psi(x_d)), \quad \Psi:[0,1] \mapsto [0,1],
\end{equation*}
which allows us to write $\mu$ in the form of \eqref{eqn:defn_mu} with
\begin{equation*}
f(\vx) = g(\vPsi(\vx)) \prod_{\ell = 1}^d \Psi'(x_l).
\end{equation*}
\end{subequations}
If $\Psi$ is sufficiently smooth, $\lim_{x \downarrow 0}x^{-r}\Psi'(x) = \lim_{x \uparrow 1} (x-1)^{-r}\Psi'(x) = 0$ for $ r \in \natzero$, and $g \in C^{(r, \ldots, r)}[0,1]^d$, then $f$ has continuous, periodic derivatives up to order $r$ in each direction.  
Examples of this kind of transform include \cite{Sid08a}:
\begin{align*}
%C^0 &: \Psi(x) =  3 x^2 - 2 x^3, \quad   \Psi'(x) = 6x(1-x), \\
%C^1 & : \Psi(x) = x^3(10-15x+6x^2),  \\
%&\qquad \qquad \qquad   \Psi'(x) = 30x^2(1-x)^2 \\
\text{Sidi's } C^1 & : \Psi(x) = x - \frac{\sin(2\pi x)}{2 \pi}, \\
&\qquad \qquad \qquad   \Psi'(x) = 1 - \cos(2\pi x), \\
\text{Sidi's } C^2 & : \Psi(x) = \frac {8 - 9 \cos(\pi x) + \cos(3 \pi x)}{16} ,  \\
&\qquad \qquad \Psi'(x) = \frac {3 \pi[3 \sin(\pi x) - \sin(3 \pi x)]}{16}.
\end{align*}

Periodizing variable transforms are used for the numerical examples below. In some cases, they can speed the convergence of the Bayesian cubature. 
	



\subsection{Test Results and Observations}
%\label{sec:numerical_experiments}

Three integrals were evaluated using the GAIL algorithm \code{cubBayesLattice\_g}:  a multivariate normal probability, the Keister's example, and an option pricing example.  
The sequences $\{\vx_i\}_{i=1}^\infty$ were the randomly shifted lattice node sequences supplied by GAIL. 
For each integral,  each of our stopping criteria---empirical Bayes, full Bayes, and generalized cross-validation---our algorithm was run for $400$ different randomly chosen error tolerances. The error tolerances were chosen randomly in an interval depending on the difficulty of the problem. In each run, the nodes were also randomly shifted. The accuracy of the algorithm differs based on the shift.
For each test, the execution times were plotted against $\abs{\mu - \hmu}/\varepsilon$.  We expect $\abs{\mu - \hmu}/\varepsilon$ to be no greater than one, but hope that it is not too much smaller than one, which would indicate a stopping criterion that is too conservative. 

Figures \ref{fig:mvn-guaranteed-MLE} to \ref{fig:optprice-guaranteed-GCV} can be reproduced using the script \code{cubBayesLattice\_guaranteed\_plots.m} in GAIL.


\paragraph{Multivariate Normal Probability.}

This example was already introduced in Section \ref{MVN_example}, where we used the Mat\'ern covariance kernel.  Here we apply Sidi's $C^2$  periodization to $ f_{\textup{Genz}}$ \eqref{fGenzdef}, choose $d=3$ and $r=2$. The simulation results for this example function are summarized in Figures \ref{fig:mvn-guaranteed-MLE}, \ref{fig:mvn-guaranteed-FB}, and \ref{fig:mvn-guaranteed-GCV}.  In all cases, the Bayesian cubature returns an approximation within the prescribed error tolerance. We used the same setting as before with generic slow Bayesian cubature in \secref{MVN_example} for comparison. For error threshold $\varepsilon=10^{-5}$ with the empirical Bayes stopping criterion, our fast algorithm takes just under 0.01 second as shown in \figref{fig:mvn-guaranteed-MLE} whereas the basic algorithm takes over 20 seconds as shown in \figref{fig:MVN_Metern_d2b2}. 

% breaking it a paragraph here, otherwise it looks big
Amongst the three stopping criteria, GCV achieves the desired tolerance faster than the others. 
One can also observe from the figures, the credible intervals are in general much wider than the true error.
This could be due to the periodized integrand being smoother than the $r=2$ kernel assumes. Perhaps one should consider smoother covariance kernels.

\begin{figure}
	\centering
	%d=3 problem transformed into d=2
	\includegraphics[width=0.98\linewidth]{"Lattice_MVN_guaranteed_time_MLE_C2sin_d2_r2_2019-Jun-27"}
	\caption[Guaranteed:]{Multivariate normal probability example using the empirical Bayes stopping criterion.}
	\label{fig:mvn-guaranteed-MLE}
	\centering
	\includegraphics[width=0.98\linewidth]{"Lattice_MVN_guaranteed_time_full_C2sin_d2_r2_2019-Jun-27"}
	\caption[MVN guaranteed : FB]{Multivariate normal probability example using the full Bayes stopping criterion.}
	\label{fig:mvn-guaranteed-FB}
	\centering
	\includegraphics[width=0.98\linewidth]{"Lattice_MVN_guaranteed_time_GCV_C2sin_d2_r2_2019-Jun-27"}
	\caption[MVN guaranteed : GCV]{Multivariate normal probability example using the GCV stopping criterion.}
	\label{fig:mvn-guaranteed-GCV}
\end{figure}

\paragraph{Keister's Example.}

This multidimensional integral function comes from \cite{Kei96} and is inspired by a physics application:
\begin{align*}
\mu & =  \int_{\reals^d} \cos(\norm{ \vt}) \exp(-\norm{ \vt }^2) \, \dvt \\
\nonumber
%&  = 
%\int_{\reals^d} \cos(a\norm{ \vt}) \exp(-a^2\norm{ \vt }^2)  a^d \, \dvt \\
& = \int_{[0,1]^d} f_{\textup{Keister}}(\vx) \, \dvx,\\
\intertext{where }
f_\textup{Keister}(\vx) &= \pi^{d/2} \cos\left(\norm{ \Phi^{-1}(\vx)/2}\right)  ,
\end{align*}
and again $\Phi$ is the standard normal distribution.
The true value of $\mu$ can be calculated iteratively in terms of a quadrature as follows:  
\begin{equation*}
\mu = \frac{2 \pi^{d/2} I_c(d)}{\Gamma(d/2)}, \quad d=1,2, \ldots
\end{equation*}
where $\Gamma$ denotes the gamma function, and
\begin{align*}
I_c(1) &= \frac{\sqrt{\pi}}{2 \exp(1/4)}, 
\\
I_s(1) &= \int_{x=0}^\infty \exp(-\vx^T\vx)\sin(\vx) \, \dvx 
\\
& =  0.4244363835020225,
\\
I_c(2) &= \frac{1-I_s(1)}{2}, \qquad
I_s(2) = \frac{I_c(1)}{2}
\\
I_c(j) &= \frac{(j-2)I_c(j-2)-I_s(j-1)}{2},
\qquad j =3,4,\ldots
\\
I_s(j) &= \frac{(j-2)I_s(j-2)-I_c(j-1)}{2},
\qquad j =3,4,\ldots.
% ref: https://www.mathworks.com/help/matlab/ref/gamma.html
\end{align*}
\begin{figure}
	\centering
	\includegraphics[width=0.95\linewidth]{"Lattice_Keister_guaranteed_time_MLE_C1sin_d4_r2_2019-Jun-27"}
	\caption[Keister guaranteed:MLE]{Keister example using the empirical Bayes stopping criterion.}
	\label{fig:keister-guaranteed-MLE}
%\end{figure}
%\begin{figure}
	\centering
	\includegraphics[width=0.95\linewidth]{"Lattice_Keister_guaranteed_time_full_C1sin_d4_r2_2019-Jun-27"}
	\caption[Keister guaranteed:FB]{Keister example using the full Bayes stopping criterion.}
	\label{fig:keister-guaranteed-FB}
%\end{figure}
%\begin{figure}
	\centering
	\includegraphics[width=0.95\linewidth]{"Lattice_Keister_guaranteed_time_GCV_C1sin_d4_r2_2019-Jun-27"}
	\caption[Keister guaranteed:GCV]{Keister example using the GCV stopping criterion.}
	\label{fig:keister-guaranteed-GCV}
\end{figure}

Figures \ref{fig:keister-guaranteed-MLE}, \ref{fig:keister-guaranteed-FB} and \ref{fig:keister-guaranteed-GCV} summarize the numerical tests for this integral.  We used the Sidi's $C^1$ periodization, dimension $d=4$, and $r=2$. 
As we can see, the GCV stopping criterion achieves faster results than the other stopping criteria, similarly to the multivariate normal case.

\paragraph{Option Pricing.}

The price of financial derivatives can often be modeled by high dimensional integrals. If the underlying asset is described in terms of a discretized geometric Brownian motion, then the fair price of the option is:
\begin{equation*}
\mu = \int_{\reals^d} \text{payoff}(\vz) \frac{\exp(\frac 12 \vz^T\mSigma^{-1}\vz)}{\sqrt{(2\pi)^d \det(\mSigma)}} \, \dvz = \int_{[0,1]^d} f(\vx) \, \dvx,
\end{equation*} 
where {payoff($\cdot$)} defines the discounted payoff of the option,
\begin{align*}
\mSigma &= (T/d) \bigl(\min(j,k) \bigr)_{j,k=1}^d = \mL \mL^T,\\
f(\vx) &= \text{payoff} \left(\mL 
\begin{pmatrix}
\Phi^{-1}(x_1) \\ \vdots \\ \Phi^{-1}(x_d)
\end{pmatrix} \right).
\end{align*}
The Asian arithmetic mean call option has a payoff of the form
\begin{align*}
\text{payoff}(\vz) &= \max\left( \frac 1d  \sum_{j=1}^d S_j(\vz) - K, 0 \right) \me^{-\upsilon T}, \\
\text{where}\;
S_j(\vz) &= S_0 \exp\bigl((\upsilon-\sigma^2/2)jT/d + \sigma \sqrt{T/d} z_j \bigr).
\end{align*}
Here, $T$ denotes the time to maturity of the option, $d$ the number of time steps, $S_0$ the initial price of the stock, $\upsilon$ the interest rate, $\sigma$ the volatility, and $K$ the strike price.  

Figures \ref{fig:optprice-guaranteed-MLE}, \ref{fig:optprice-guaranteed-FB} and 
\ref{fig:optprice-guaranteed-GCV} summarize the numerical results for this example using
$
T = 1/4, \ \ d = 13, \ \ S_0 = 100, \ \ \upsilon =  0.05, \ \ \sigma = 0.5, \ \ K = 100.
$
Moreover, $\mL$ is chosen to be the matrix of eigenvectors of $\mSigma$ times the square root of the diagonal matrix of eigenvalues of $\mSigma$.
Because the integrand has a kink caused by the $\max$ function, it does not help to use a periodizing transform that is very smooth.  We choose the Baker's transform \eqref{eq:bakerTrans} and $r = 1$.

\begin{figure}
	\centering
	\includegraphics[width=0.95\linewidth]{"Lattice_optPrice_guaranteed_time_MLE_Baker_d12_r1_2019-Jul-9"}
	\caption[Option pricing Guaranteed: MLE]{Option pricing using the empirical Bayes stopping criterion.}
	\label{fig:optprice-guaranteed-MLE}
%\end{figure}
%\begin{figure}
	\centering
	\includegraphics[width=0.95\linewidth]{"Lattice_optPrice_guaranteed_time_full_Baker_d12_r1_2019-Jul-9"}
	\caption[OptPrice guaranteed : FB]{Option pricing using the full Bayes stopping criterion.}
	\label{fig:optprice-guaranteed-FB}
%\end{figure}
%\begin{figure}
	\centering
	\includegraphics[width=0.95\linewidth]{"Lattice_optPrice_guaranteed_time_GCV_Baker_d12_r1_2019-Jul-8"}
	\caption[OptPrice guaranteed : GCV]{Option pricing using the  GCV stopping criterion.}
	\label{fig:optprice-guaranteed-GCV}
\end{figure}


In summary, the Bayesian cubature algorithm computes the integral within the user-specified threshold in nearly all of the test cases.  The rare exceptions occurred in the option pricing example for $\varepsilon=10^{-4}$. Our algorithm used the maximum allowed sample size and still did not reach the stopping criterion $\err_{\textup{CI}} \leq \varepsilon$, due to the complexity and high dimension of the integrand. 
%Also notice, our algorithm finished within 10 seconds for Keister and multivariate Normal examples while the option pricing took closer to 70 seconds, again due to the complexity of the integrand.
%FH it is difficult to compare apples to orances, and the timing depends on the error tolerances as well

A noticeable aspect from the plots is how much the error bounds differ from the true error. For option pricing example, the error bound is not as conservative as it is for the multivariate normal and Keister examples. A possible reason is that the latter integrands are significantly smoother than the covariance kernel assumed.  This is a matter for further investigation.


\iffalse
\subsection{Diagnostics for the Gaussian Process Assumption}


The starting point for our Bayesian cubature is the assumption that the integrand arises from a Gaussian process. This means that the function data, $\vf$ satisfy a multivariate Gaussian distribution, as in \eqref{eqn:fGaussDist}.  The transformed data, $\vZ = ( n\mLambda)^{-\frac 12} \mV^H(\vf - m \vone)$ has zero mean and is also uncorrelated because
\begin{align*}
\cov (\vZ) 
&= \frac 1n \Ex\left[  
\mLambda^{-\frac 12} \mV^H (\vf - m \vone)
(\vf - m \vone)^T \mV \mLambda^{-\frac 12}
\right]
\\
&=
 \mLambda^{-\frac 12} \mV^H 
\Ex\left[ (\vf - m \vone)
(\vf - m \vone)^T \right] \mV \mLambda^{-\frac 12}
\\
&=
\frac{1}{n} \mLambda^{-\frac 12} \mV^H 
\frac 1n \mV \mLambda \mV^H \mV \mLambda^{-\frac 12}
 = \mathsf{I}
\end{align*}
Thus, the elements of $\vZ$ are IID standard Gaussian random variables.  

\figref{fig:mvn-normplot} and \figref{fig:keister-normplot} are normal probability plots of the $Z_i$ using empirical Bayes estimates of $m$ and $\vtheta$. \textbf{more goes here}.



\begin{figure}
	\centering
	\includegraphics[width=0.9\linewidth]{"figures/arbMean/Keister/C1sin/Keister Normplot d_2 bernoulli_2 Period_C1sin n_32768"}
	\caption{Normal plot : Keister function with arbMean assumption}
	\label{fig:keister-normplot}
\end{figure}




\begin{figure}
	\centering
	\includegraphics[width=0.9\linewidth]{"figures/arbMean/MVN/C1sin/MVN Normplot d_2 bernoulli_2 Period_C1sin n_32768"}
	\caption{Normal plot : MVN with arbMean assumption}
	\label{fig:mvn-normplot}
\end{figure}

\fi




\section{Discussion and Further Work}

We have developed a fast, automatic Bayesian cubature that estimates a multidimensional definite integral within a user defined error tolerance.  The stopping criteria arise from assuming the integrand to be a Gaussian process.  There are three approaches:  empirical Bayes, full Bayes, and generalized cross-validation.  The computational cost of the automatic Bayesian cubature can be dramatically reduced if the covariance kernel matches the nodes.  One such match in practice is rank-1 lattice nodes and shift-invariant kernels.  The matrix-vector multiplications can be accomplished using the fast Fourier Transform.  The performance of our automatic Bayesian cubature are illustrated using three integration problems.  

Digital sequences and digital shift and/or scramble invariant kernels have the potential of being another match that satisfies the conditions in Section \ref{sec:fast_BC}.  The fast transform would correspond to a fast Walsh transform.  

%For such kernels, periodicity is not assumed; however, special structure of both the sequences and the kernels are required to take advantage of integrand smoothness.

One should be able to adapt our Bayesian cubature to control variates, i.e., assuming  
\begin{equation*}
f = \mathcal{GP} \left( \beta_0 + \beta_1 g_1 + \cdots + \beta_p g_p, s^2 C \right),
\end{equation*}
for some choice of $g_1, \ldots, g_p$ whose integrals are known, and some parameters $\beta_0, \ldots, \beta_p$ in addition to $s$ and $C$.  The efficacy of this approach has not yet been explored.




%\section{Appendix}

%\iffalse
%\subsection{Properties of Multivariate Normal Distributions}




%\fi






\iffalse

% For one-column wide figures use
\begin{figure}
% Use the relevant command to insert your figure file.
% For example, with the graphicx package use
  \includegraphics{example.eps}
% figure caption is below the figure
\caption{Please write your figure caption here}
\label{fig:1}       % Give a unique label
\end{figure}
%
% For two-column wide figures use
\begin{figure*}
% Use the relevant command to insert your figure file.
% For example, with the graphicx package use
  \includegraphics[width=0.75\textwidth]{example.eps}
% figure caption is below the figure
\caption{Please write your figure caption here}
\label{fig:2}       % Give a unique label
\end{figure*}
%
% For tables use
\begin{table}
% table caption is above the table
\caption{Please write your table caption here}
\label{tab:1}       % Give a unique label
% For LaTeX tables use
\begin{tabular}{lll}
\hline\noalign{\smallskip}
first & second & third  \\
\noalign{\smallskip}\hline\noalign{\smallskip}
number & number & number \\
number & number & number \\
\noalign{\smallskip}\hline
\end{tabular}
\end{table}

\fi



\begin{acknowledgements}
This research was supported in part by the National Science Foundation grants DMS-1522687 and DMS-1638521 (SAMSI).
The authors would like to thank the organizers of the SAMSI-Lloyds-Turing Workshop on Probabilistic Numerical Methods, where a preliminary version of this work was discussed.  The authors also thank Chris Oates and Sou-Cheng Choi for valuable comments.
\end{acknowledgements}

% BibTeX users please use one of
%\bibliographystyle{spbasic}      % basic style, author-year citations
%\bibliographystyle{spmpsci}      % mathematics and physical sciences
%\bibliographystyle{spphys}       % APS-like style for physics
%\bibliography{}   % name your BibTeX data base
\bibliography{FJHown23,FJH23}
\bibliographystyle{spmpsci}


\begin{appendices}
\section{Details of the Full Bayes Posterior Density for $\mu$} \label{appendix:full_bayes}
Starting from the Bayesian formula for the posterior density for $\mu$ at the beginning of Section \ref{sec:fullBayes} with the non-informative prior, it follows that 
\begin{align*}
\MoveEqLeft[1]{\rho_{\mu|\vf}(z | \vy)}\\
& \propto \int_{0}^\infty \int_{-\infty}^\infty 
\rho_{\mu | m, s^2, \vf}(z | \xi, \lambda , \vy) \\
& \qquad \times  \rho_{\vf | m, s^2} (\vy | \xi, \lambda) \, \vrho_{m,s^2}(\xi,\lambda)  \, \D \xi \D \lambda \\
& \propto \displaystyle \int_{0}^\infty  \frac{1}{\lambda^{(n+3)/2}} \cdots \\
& \cdots \int_{-\infty}^\infty  \exp \biggl( -\frac{1}{2\lambda}\biggl\{
\frac{
	[z - \xi (1 - \vc^T \mC^{-1} \vone)  -  \vc^T \mC^{-1} \vy]^2}
{c_0  -\vc ^T \mC^{-1} \vc}  \\
& \qquad + (\vy - \xi \vone)^T \mC^{-1}(\vy - \xi \vone) \biggr \} \biggr) \, \D \xi \D \lambda \\
& \qquad \qquad
\text{by \eqref{eqn:fGaussDist}, \eqref{eqn:condInteg}} \; \text{and} \; \rho_{m,s^2}(\xi,\lambda) \propto 1/\lambda \\
& \propto \displaystyle \int_{0}^\infty  \frac{1}{\lambda^{(n+3)/2}}  \cdots  \\ 
& \qquad  \cdots  \int_{-\infty}^\infty  \exp\left( -\frac{\alpha \xi^2 -2 \beta \xi + \gamma}{2\lambda(c_0  -\vc ^T \mC^{-1} \vc)} \right) \, \D \xi \D \lambda,
\end{align*}
where
\begin{align*}
\alpha & = (1 - \vc^T \mC^{-1} \vone)^2 + \vone^T \mC^{-1} \vone (c_0  -\vc ^T \mC^{-1} \vc),\\
\beta & =(1 - \vc^T \mC^{-1} \vone)(z - \vc^T \mC^{-1} \vy ) \\
& \qquad \qquad  + \vone^T \mC^{-1} \vy (c_0  -\vc ^T \mC^{-1} \vc),\\
\gamma &  = (z - \vc^T \mC^{-1} \vy )^2  + \vy^T \mC^{-1} \vy (c_0  -\vc ^T \mC^{-1} \vc).
\end{align*}
In the derivation above and below, factors that are independent of $\xi$, $\lambda$, or $z$ can be discarded since we only need to preserve the proportion.  But, factors that depend on $\xi$, $\lambda$, or $z$ must be kept.  
%Completing the square allows us to compute the integral with respect to $\xi$:
Completing the square, $
\alpha \xi^2 -2 \beta \xi + \gamma 
= \alpha (\xi -\beta/\alpha)^2  - (\beta^2/\alpha) + \gamma,
$
allows us to evaluate the integrals with respect to $\xi$ and $\lambda$:
\begin{align*}
\MoveEqLeft{\rho_{\mu|\vf}(z | \vy)} \\
& \propto \displaystyle \int_{0}^\infty  \frac{1}{\lambda^{(n+3)/2}}  \exp\left( -\frac{  \gamma - \beta^2/\alpha}{2\lambda(c_0  -\vc ^T \mC^{-1} \vc)} \right)  \cdots \\
& \qquad \qquad \cdots \int_{-\infty}^\infty  \exp\left( -\frac{\alpha (\xi -\beta/\alpha)^2}{2\lambda(c_0  -\vc ^T \mC^{-1} \vc)} \right) \, \D \xi \D \lambda \\
& \propto \displaystyle \int_{0}^\infty  \frac{1}{\lambda^{(n+2)/2}}  \exp\left( -\frac{  \gamma - \beta^2/\alpha}{2\lambda(c_0  -\vc ^T \mC^{-1} \vc)} \right) \D \lambda \\
& \propto \left(\gamma - \frac{\beta^2}{\alpha}\right)^{-n/2} \propto \left(\alpha \gamma - \beta^2\right)^{-n/2}.
\end{align*}
Finally, we simplify the key term via straightforward calculations to the following:
\begin{align*}
\alpha \gamma - \beta^2 \propto 1 +  \frac{(z - \hmu_{\textup{MLE}})^2}{(n-1)s_{\textup{full}}^2},
\end{align*}
where 
\begin{multline*}
\hsigma_{\textup{full}}^2
:= \frac{1}{n-1}
\vy^T\left[ \mC^{-1} 
- \frac{ \mC^{-1} \vone\vone^T \mC^{-1}}{\vone^T \mC^{-1} \vone}  \right]\vy
\\ 
\times  \left[\frac{(1 - \vc^T \mC^{-1} \vone)^2}{\vone^T \mC^{-1} \vone} + (c_0  -\vc ^T \mC^{-1} \vc) \right].
\end{multline*}
This completes the derivation of \eqref{eqn:fullBayesrho}.
\end{appendices}








\iffalse
% Non-BibTeX users please use
\begin{thebibliography}{}
%
% and use \bibitem to create references. Consult the Instructions
% for authors for reference list style.
%
\bibitem{RefJ}
% Format for Journal Reference
Author, Article title, Journal, Volume, page numbers (year)
% Format for books
\bibitem{RefB}
Author, Book title, page numbers. Publisher, place (year)
% etc
\end{thebibliography}
\fi

\end{document}
% end of file template.tex